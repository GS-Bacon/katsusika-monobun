\documentclass[
a4paper,
10pt,
book]
{tarticle}
% A4 20pt book tarticle→縦書き
\usepackage{pxrubrica} %ルビ追加
\usepackage[deluxe]{otf}
\usepackage{comment} %ブロックでコメントアウト


\AtBeginDvi{\special{papersize=210mm,297mm}} %A4サイズ 210x297

\setlength{\topmargin}{24mm} % ページ上部余白の設定
\addtolength{\topmargin}{-1in} % 初期設定の1インチ分を引いておく。

\setlength{\oddsidemargin}{22mm} % 同、奇数ページ左。
\addtolength{\oddsidemargin}{-1in}

\setlength{\evensidemargin}{28mm} % 同、偶数ページ左。
\addtolength{\evensidemargin}{-1in}

\setlength{\textwidth}{245mm} % 文書領域の幅(上下)。縦書と横書でパラメータ(width / height)の向きが変わる。

\setlength{\textheight}{160mm} % 文書領域の幅(左右)。
\setlength{\headsep}{1mm} % ヘッダ下端と本文領域との間隔。私の場合、ヘッダをつけぬので、この辺の数値は0である。しかし headsep って何や。headseparation? headseparator?
\setlength{\headheight}{0mm} % headheight → ヘッダの高さ。ヘッダはつけないが、設定しておかぬと TeX の自動配置の際にデフォルト値(知らんけど)を入れられそう。
\setlength{\topskip}{0mm} % ここは改善の余地あり→下段 [参考] "行位置が揃った段組" を見ること。
\setlength{\parskip}{0pt} % 段落間の自動空白調整を切る。parskip って何の略や。parindent っちゅうのもあるけど、par って何なんや? → par とは paragraph(段落)のことらしい。 parindent は、1行目の字下げ幅。

%1ページあたりの行数、文字数を設定するためのなんかよくわからんやつ
\makeatletter
\def\mojiparline#1{
    \newcounter{mpl}
    \setcounter{mpl}{#1}
    \@tempdima=\linewidth
    \advance\@tempdima by-\value{mpl}zw
    \addtocounter{mpl}{-1}
    \divide\@tempdima by \value{mpl}
    \advance\kanjiskip by\@tempdima
    \advance\parindent by\@tempdima
    }
    \makeatother
    \def\linesparpage#1{
        \baselineskip=\textheight
        \divide\baselineskip by #1
        }
        \begin{document}
\Large
%normalsizeを10ptに設定して14pt

% 一行あたり文字数の指定
\mojiparline{50}
% 1ページあたり行数の指定
\linesparpage{20}

%\fontsize{16pt}{28pt}\selectfont
\rubysetup{(m)} %ルビのデフォルト設定 モノルビで
\mcfamily %フォント設定 ヒラノギ細字を使う

%表紙
\thispagestyle{empty}
{\fontsize{30pt}{24pt}\selectfont
クねずみ\\
\begin{flushright}
    \ruby{宮沢}{みや|ざわ}\ruby{賢治}{けん|じ}\\
\end{flushright}
}
\newpage
\setcounter{page}{1}

\indent クという名前のねずみがありました。たいへん\ruby{高慢}{ごう|まん}でそれにそねみ深くって、
自分をねずみの仲間の一番の学者と思っていました。ほかのねずみが何か生意気なことを言うとエヘンエヘ~ンと言うのが\ruby{癖}{くせ}でした。\\
\indent クねずみのうちへ、ある日、友だちのタねずみがやって来ました。\\
\indent さてタねずみはクねずみに言いました。\\
「\ruby[<g>]{今日}{こんにち}は、クさん。いいお天気です。」\\
「いいお天気です。何かいいものを見つけましたか。」\\
「いいえ。どうも不景気ですね。どうでしょう。これからの景気は。」\\
「さあ、あなたはどう思いますか。」\\
「そうですね。しかしだんだんよくなるのじゃないでしょうか。オウベイのキンユウはしだいにヒッパクをテイしたそう……。」\\
「エヘン、エヘン。」いきなりクねずみが大きなせきばらいをしましたので、タねずみはびっくりして飛びあがりました。クねずみは横を向いたまま、
ひげを一つぴんとひねって、それから口の中で、\\
「ヘイ、それから。」と言いました。\\
\indent タねずみはやっと安心してまたおひざに手を置いてすわりました。\\
\indent クねずみもやっとまっすぐを向いて言いました。\\
「\ruby{先}{せん}ころの地震にはおどろきましたね。」\\
「全くです。」\\
「あんな大きいのは私もはじめてですよ。」\\
「ええ、ジョウカドウでしたねえ。シンゲンはなんでもトウケイ四十二度二分ナンイ……。」\\
「エヘン、エヘン。」\\
\indent クねずみはまたどなりました。\\
\indent タねずみはまた\ruby{面}{めん}くらいましたが、さっきほどではありませんでした。\\
\indent クねずみはやっと気を直して言いました。\\
「天気もよくなりましたね。あなたは何かうまい仕掛けをしておきましたか。」\\
「いいえ、なんにもしておきません。しかし、今度天気が長くつづいたら、私は少し畑の方へ出てみようと思うんです。」\\
「畑には何かいいことがありますか。」\\
「秋ですからとにかく何かこぼれているだろうと思います。天気さえよければいいのですがね。」\\
「どうでしょう。天気はいいでしょうか。」\\
「そうですね、新聞に出ていましたが、オキナワレットウにハッセイしたテイキアツは次第にホクホクセイのほうへシンコウ……。」\\
「エヘン、エヘン。」クねずみはまたいやなせきばらいをやりましたので、タねずみはこんどというこんどはすっかりびっくりして半分立ちあがって、
ぶるぶるふるえて目をパチパチさせて、黙りこんでしまいました。\\
\indent クねずみは横の方を向いて、おひげをひっぱりながら、横目でタねずみの顔を見ていましたが、ずうっとしばらくたってから、
あらんかぎり声をひくくして、\\
「へい。そして。」と言いました。ところがタねずみはもうすっかりこわくなって物が言えませんでしたから、
にわかに一つていねいにおじぎをしました。そしてまるで細いかすれた声で、\\
「さよなら。」と言ってクねずみのおうちを出て行きました。\\
\indent クねずみは、そこであおむけにねころんで、\\
「ねずみ競争新聞」を手にとってひろげながら、\\
「ヘッ。タなどはなってないんだ。」とひとりごとを言いました。\\
\indent さて、「ねずみ競争新聞」というのは実にいい新聞です。これを読むと、ねずみ仲間の競争のことはなんでもわかるのでした。
ペねずみが、たくさんとうもろこしのつぶをぬすみためて、大砂糖持ちのパねずみと意地ばりの競争をしていることでも、
ハねずみヒねずみフねずみの三匹のむすめねずみが学問の競争をやって、比例の問題まで来たとき、
とうとう三匹とも頭がペチンと裂けたことでも、なんでもすっかり出ているのでした。\\
\indent さあ、さあ、みなさん。失礼ですが、クねずみのきょうの新聞を読むのを、お聞きなさい。\\
「ええと、カマジン国の飛行機、プハラを襲うと。なるほどえらいね。これはたいへんだ。まあしかし、
ここまでは来ないから大丈夫だ。ええと、ツェねずみの行くえ不明。ツェねずみというのはあの意地わるだな。こいつはおもしろい。\\
\indent 天井裏街一番地、ツェ氏は昨夜行くえ不明となりたり。本社のいちはやく探知するところによれば
ツェ氏は数日前より\textbf{はりがねせい}、\textbf{ねずみとり}氏と交際を結びおりしが
一昨夜に至りて両氏の間に多少感情の衝突ありたるもののごとし。台所街四番地ネ氏の談によれば昨夜もツェ氏は、
\textbf{はりがねせい}、\textbf{ねずみとり}氏を訪問したるがごとし、
と。なお床下通り二十九番地ポ氏は、昨夜深更より今朝にかけて、ツェ氏並びに\textbf{はりがねせい}、
\textbf{ねずみとり}氏の激しき争論、時に格闘の声を聞きたりと。
以上を総合するに、本事件には、\textbf{はりがねせい}、
\textbf{ねずみとり}氏、最も深き関係を有するがごとし。本社はさらに深く事件の真相を探知の上、
大いに\textbf{はりがねせい}、\textbf{ねずみとり}氏に\ruby[(g)]{筆誅}{ひっちゅう}を加えんと欲す。
と。ははは、ふん、これはもう疑いもない。ツェのやつめ、ねずみとりに食われたんだ。おもしろい。
そのつぎはと。なんだ、ええと、新任ねずみ会議員テ氏。エヘン、エヘン。エン。エッヘン。ヴェイヴェイ。
なんだちくしょう。テなどがねずみ会議員だなんて。えい、おもしろくない。おれでもすればいいんだ。えい。おもしろくもない、散歩に出よう。」\\
\indent そこでクねずみは散歩に出ました。そしてプンプンおこりながら、天井裏街の方へ行く途中で、
二匹のむかでが親孝行の\ruby[<g>]{蜘蛛}{くも}の話をしているのを聞きました。\indent \\
「ほんとうにね、そうはできないもんだよ。」\\
「ええ、ええ、全くですよ。それにあの子は、自分もどこかからだが悪いんですよ。
それだのにね、朝は二時ごろから起きて薬を飲ませたり、おかゆをたいてやったり、
夜だって寝るのはいつもおそいでしょう。たいてい三時ごろでしょう。ほんとうにからだがやすまるってないんでしょう。感心ですねえ。」\\
「ほんとうにあんな心がけのいい子は今ごろあり……。」\\
「エヘン、エヘン。」と、いきなりクねずみはどなって、おひげを横の方へひっぱりました。\\
\indent むかではびっくりして、はなしもなにもそこそこに別れて逃げて行ってしまいました。\\
\indent クねずみはそれからだんだん天井裏街の方へのぼって行きました。天井裏街のガランとした広い通りでは、
ねずみ会議員のテねずみがもう一ぴきのねずみとはなしていました。\\
\indent クねずみはこわれたちり取りのかげで立ちぎきをしておりました。\\
\indent テねずみが、\\
「それで、その、わたしの考えではね、どうしてもこれは、その、共同一致、団結、\ruby{和睦}{わ|ぼく}の、セイシンで、
やらんと、いかんね。」と言いました。\\
\indent クねずみは、\\
「エヘン、エヘン。」と聞こえないようにせきばらいをしました。相手のねずみは、「へい。」と言って考えているようです。\\
\indent テねずみははなしをつづけました。\\
「もしそうでないとすると、つまりその、世界のシンポハッタツ、カイゼンカイリョウがそのつまりテイタイするね。」\\
「エン、エン、エイ、エイ。」クねずみはまたひくくせきばらいをしました。\\
\indent 相手のねずみは、「へい。」と言って考えています。\\
「そこで、その、世界文明のシンポハッタツ、カイリョウカイゼンがテイタイすると、政治はもちろんケイザイ、
ノウギョウ、ジツギョウ、コウギョウ、キョウイク、ビジュツそれからチョウコク、カイガ、それからブンガク、シバイ、
ええと、エンゲキ、ゲイジュツ、ゴラク、そのほかタイイクなどが、ハッハッハ、たいへんそのどうもわるくなるね。」
テねずみはむつかしいことをあまりたくさん言ったので、もう愉快でたまらないようでした。クねずみはそれがまたむやみにしゃくにさわって、
「エン、エン。」と聞こえないように、そしてできるだけ高くせきばらいをやって、にぎりこぶしをかためました。\\
\indent 相手のねずみはやはり「へい。」と言っております。\\
\indent テねずみはまたはじめました。\\
「そこでそのケイザイやゴラクが悪くなるというと、不平を生じてブンレツを起こすというケッカにホウチャクするね。
そうなるのは実にそのわれわれのシンガイでフホンイであるから、やはりその、ものごとは共同一致団結和睦のセイシンでやらんといかんね。」\\
\indent クねずみはあんまりテねずみのことばが立派で、議論がうまくできているのがしゃくにさわって、とうとうあらんかぎり、\\
「エヘン、エヘン。」とやってしまいました。するとテねずみはぶるるっとふるえて、目を閉じて、小さく小さくちぢまりましたが、
だんだんそろりそろりと延びて、そおっと目をあいて、それから大声で叫びました。\\
「こいつは、ブンレツだぞ。ブンレツ者だ。しばれ、しばれ。」と叫びました。すると相手のねずみは、
まるでつぶてのようにクねずみに飛びかかってねずみの\ruby{捕}{と}り\ruby{繩}{なわ}を出して、クルクルしばってしまいました。\\
\indent クねずみはくやしくてくやしくてなみだが出ましたが、どうしてもかないそうがありませんでしたから、
しばらくじっとしておりました。するとテねずみは紙切れを出してするするするっと何か書いて捕り手のねずみに渡しました。\\
\indent 捕り手のねずみは、しばられてごろごろころがっているクねずみの前に来て、すてきにおごそかな声でそれを読みはじめました。\\
「クねずみはブンレツ者によりて、みんなの前にて暗殺すべし。」クねずみは声をあげてチュウチュウ泣きました。\\
「さあ、ブンレツ者。あるけ、早く。」と、捕り手のねずみは言いました。
さあ、そこでクねずみはすっかり恐れ入ってしおしおと立ちあがりました。あっちからもこっちからもねずみがみんな集まって来て、\\
「どうもいい気味だね。いつでもエヘンエヘンと言ってばかりいたやつなんだ。」\\
「やっぱり分裂していたんだ。」\\
「あいつが死んだらほんとうにせいせいするだろうね。」というような声ばかりです。\\
\indent 捕り手のねずみは、いよいよ白いたすきをかけて、暗殺のしたくをはじめました。\\
\indent その時みんなのうしろの方で、フウフウと言うひどい音が聞こえ、二つの目玉が火のように光って来ました。
それは例の\ruby{猫大将}{ねこ|たい|しょう}でした。\\
「ワーッ。」とねずみはみんなちりぢり四方に逃げました。\\
「逃がさんぞ。コラッ。」と猫大将はその一匹を追いかけましたが、もうせまいすきまへずうっと深くもぐり込んでしまったので、
いくら猫大将が手をのばしてもとどきませんでした。\\
\indent 猫大将は「チェッ。」と舌打ちをして戻って来ましたが、クねずみのただ一匹しばられて残っているのを見て、びっくりして言いました。\\
「貴様はなんと言うものだ。」クねずみはもう落ち着いて答えました。\\
「クと申します。」\\
「フ、フ、そうか、なぜこんなにしているんだ。」\\
「暗殺されるためです。」\\
「フ、フ、フ。そうか。それはかあいそうだ。よしよし、おれが引き受けてやろう。おれのうちへ来い。
ちょうどおれのうちでは、子供が四人できて、それに家庭教師がなくて困っているところなんだ。来い。」\\
\indent 猫大将はのそのそ歩きだしました。\\
\indent クねずみはこわごわあとについて行きました。猫のおうちはどうもそれは立派なもんでした。
紫色の竹で編んであって中はわらや布きれでホクホクしていました。おまけにちゃあんとご飯を入れる道具さえあったのです。\\
\indent そしてその中に、\ruby{猫大将}{ねこ|たい|しょう}の子供が四人、やっと目をあいて、にゃあにゃあと鳴いておりました。\\
\indent 猫大将は子供らを一つずつなめてやってから言いました。\\
「お前たちはもう学問をしないといけない。ここへ先生をたのんで来たからな。よく習うんだよ。決して先生を食べてしまったりしてはいかんぞ。」\\
\indent 子供らはよろこんでニヤニヤ笑って口々に、\\
「おとうさん、ありがとう。きっと習うよ。先生を食べてしまったりしないよ。」と言いました。\\
\indent クねずみはどうも思わず足がブルブルしました。\\
\indent 猫大将が言いました。\\
「教えてやってくれ。おもに算術をな。」\\
「へい。しょう、しょう、承知いたしました。」とクねずみが答えました。\\
\indent 猫大将はきげんよくニャーと鳴いてするりと向こうへ行ってしまいました。\\
\indent 子供らが叫びました。\\
「先生、早く算術を教えてください。先生。早く。」\\
\indent クねずみはさあ、これはいよいよ教えないといかんと思いましたので、口早に言いました。\\
「一に一をたすと二です。」\\
「そうだよ。」子供らが言いました。\\
「一から一を引くとなんにもなくなります。」\\
「わかったよ。」\\
\indent 子供らが叫びました。\\
「一に一をかけると一です。」\\
「きまってるよ。」と猫の子供らが目をりんと張ったまま答えました。\\
「一を一で割ると一です。」\\
「それでいいよ。」と猫の子供らがよろこんで叫びました。そこでクねずみはすっかりのぼせてしまいました。\\
「一に二をたすと三です。」\\
「合ってるよ。」\\
「一から二を引くと……」と言おうとしてクねずみは、はっとつまってしまいました。\\
\indent すると猫の子供らは一度に叫びました。\\
「一から二は引かれないよ。」\\
\indent クねずみはあんまり猫の子供らがかしこいので、すっかりむしゃくしゃして、また早口に言いました。
そうでしょう。クねずみはいちばんはじめの一に一をたして二をおぼえるのに半年かかったのです。\\
「一に二をかけると二です。」\\
「そうともさ。」\\
「一を二で割ると……。」クねずみはまたつまってしまいました。すると猫の子供らはまた一度に声をそろえて、\\
「一割る二では半分だよ。」と叫びました。\\
\indent クねずみはあんまり\ruby{猫}{ねこ}の子供らの賢いのがしゃくにさわって、思わず「エヘン。エヘン。エイ。エイ。」\\
とやりました。すると猫の子供らは、しばらくびっくりしたように、顔を見合わせていましたが、やがてみんな一度に立ちあがって、\\
「なんだい。ねずめ、人をそねみやがったな。」と言いながらクねずみの足を一ぴきが一つずつかじりました。\\
\indent クねずみは非常にあわててばたばたして、急いで「エヘン、エヘン、エイ、エイ。」とやりましたがもういけませんでした。\\
\indent クねずみはだんだん四方の足から食われて行って、とうとうおしまいに四ひきの子猫は、
クねずみの胃の\ruby{腑}{ふ}のところで頭をコツンとぶっつけました。\\
\indent そこへ猫大将が帰って来て、\\
「何か習ったか。」とききました。\\
「ねずみをとることです。」と四ひきがいっしょに答えました。\\
\newpage
底本:「童話集 銀河鉄道の夜 他十四編」谷川徹三編、岩波文庫、岩波書店
\\
\indent 1951(昭和26)年10月25日第1刷発行\\
\indent 1966(昭和41)年7月16日第18刷改版発行\\
\indent 2000(平成12)年5月25日第71刷発行\\
底本の親本:「宮沢賢治全集 第八巻」筑摩書房\\
\indent 1956(昭和31)年10月\\
入力:のぶ\\
校正:鈴木厚司\\
2003年8月3日作成\\
2008年2月29日修正\\
青空文庫作成ファイル:\\
このファイルは、インターネットの図書館、"http://www.aozora.gr.jp/">青空文庫(http://www.aozora.gr.jp/)で作られました。入力、校正、制作にあたったのは、ボランティアの皆さんです。\\
\\
\\

\begin{comment}
\end{comment}
\end{document}