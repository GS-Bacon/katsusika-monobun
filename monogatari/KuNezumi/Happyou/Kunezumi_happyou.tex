\documentclass[
    a4paper,
    10pt,
    book]
    {tarticle}
\usepackage{daihon}
\usepackage{setspace}
\begin{document}
\Large

%\fontsize{16pt}{28pt}\selectfont

%表紙
\thispagestyle{empty}
{\fontsize{30pt}{24pt}\selectfont
    クねずみ\\
    \begin{flushright}
        \ruby{宮沢}{みや|ざわ}\ruby{賢治}{けん|じ}\\
    \end{flushright}
}
\newpage
\setcounter{page}{1}
\pagewiselinenumbers
\thispagestyle{fancy}
\fancyhead[R]
{
    \begin{spacing}{0.6}
        \fontsize{11}{20pt}\selectfont
        {
            P1\\
            1G:言\\
            2G:視\\
            3G:聴\\
            \indent\\
        }
    \end{spacing}
    \makebox[-80mm][r]{
        \parbox<t>[b]{5zw}{\fontsize{10}{25.5pt}\selectfont{
                \vskip.15\baselineskip
                \indent\\
                \indent\\
                \indent\\
                \indent\\
                \indent\\
                \indent\\
                \indent\\
                \indent\\
                \indent\\
                \indent\\
                \indent\\
                \indent\\
                \indent\\
                \indent\\
                \indent\\
                \indent\\
                \indent\\
                \indent\\
            }}}\par
}
\indent クという\ruby{名前}{な|まえ}のねずみがありました。たいへん\ruby{高慢}{ごう|まん}でそれにそねみ\ruby{深}{ぶか}くって、
\ruby{自分}{じ|ぶん}をねずみの\ruby{仲間}{なか|ま}の一\ruby{番}{ばん}の\ruby{学者}{がく|しゃ}と\ruby{思}{おも}っていました。
ほかのねずみが\ruby{何}{なに}か\ruby[<g>]{生意気}{なまいき}なこと
を\ruby{言}{い}うとエヘンエヘンと\ruby{言}{い}うのが\ruby{癖}{くせ}でした。\\
\indent クねずみのうちへ、ある\ruby{日}{ひ}、\ruby{友}{とも}だちのタねずみがやって\ruby{来}{き}ました。\\
\indent さてタねずみはクねずみに\ruby{言}{い}いました。\\
「\ruby[<g>]{今日}{こんにち}は、クさん。いいお\ruby{天気}{て|んき}です。」\\
「いいお\ruby{天気}{てん|き}です。\ruby{何}{なに}かいいものを\ruby{見}{み}つけましたか。」\\
「いいえ。どうも\ruby{不景気}{ふ|けい|き}ですね。どうでしょう。これからの\ruby{景気}{けい|き}は。」\\
「さあ、あなたはどう\ruby{思}{おも}いますか。」\\
「そうですね。しかしだんだんよくなるのじゃないでしょうか。オウベイのキンユウはしだいにヒッパクをテイしたそう……。」\\
「エヘン、エヘン。」いきなりクねずみが\ruby{大}{おお}きなせきばらいをしましたので、タねずみはびっくりして\ruby{飛}{と}びあがりました。
クねずみは\ruby{横}{よこ}を\ruby{向}{む}いたまま、ひげを\ruby{一}{ひと}つぴんとひねって、それから\ruby{口}{くち}の\ruby{中}{なか}で、\\
「ヘイ、それから。」と\ruby{言}{い}いました。\\
\indent タねずみはやっと\ruby{安心}{あん|しん}してまたおひざに\ruby{手}{て}を\ruby{置}{お}いてすわりました。\\
\indent クねずみもやっとまっすぐを\ruby{向}{む}いて\ruby{言}{い}いました。\\
「\ruby{先}{せん}ころの\ruby{地震}{じ|しん}にはおどろきましたね。」

\newpage
\thispagestyle{fancy}

\noindent \,「\ruby{全}{まった}くです。」\\
「あんな\ruby{大}{おお}きいのは\ruby{私}{わたし}もはじめてですよ。」\\
「ええ、ジョウカドウでしたねえ。シンゲンはなんでもトウケイ四十二\ruby{度}{ど}二\ruby{分}{ふん}ナンイ……。」\\
「エヘン、エヘン。」\\
\indent クねずみはまたどなりました。\\
\indent タねずみはまた\ruby{面}{めん}くらいましたが、さっきほどではありませんでした。\\
\indent クねずみはやっと\ruby{気}{き}を\ruby{直}{なお}して\ruby{言}{い}いました。\\
「\ruby{天気}{てん|き}もよくなりましたね。あなたは\ruby{何}{なに}かうまい\ruby[<g>]{仕掛}{しか}けをしておきましたか。」\\
「いいえ、なんにもしておきません。しかし、\ruby[<g>]{今度}{こんど} \ruby{天気}{てん|き}が\ruby{長}{なが}くつづいたら、\ruby{私}{わたし}は\ruby{少}{すこ}し\ruby{畑}{はたけ}の\ruby{方}{ほう}へ\ruby{出}{で}てみようと\ruby{思}{おも}うんです。」\\
「\ruby{畑}{はたけ}には\ruby{何}{なに}かいいことがありますか。」\\
「\ruby{秋}{あき}ですからとにかく\ruby{何}{なに}かこぼれているだろうと\ruby{思}{おも}います。\ruby{天気}{てん|き}さえよければいいのですがね。」\\
「どうでしょう。\ruby{天気}{てん|き}はいいでしょうか。」\\
「そうですね、\ruby{新聞}{しん|ぶん}に\ruby{出}{で}ていましたが、オキナワレットウにハッセイしたテイキアツは\ruby[<g>]{次第}{しだい}にホクホクセイのほうへシンコウ……。」\\
「エヘン、エヘン。」クねずみはまたいやなせきばらいをやりましたので、タねずみはこんどというこんどはすっかりびっくりして\ruby{半分}{はん|ぶん}\ruby{立}{た}ちあがって、ぶるぶるふるえて\ruby{目}{め}をパチパチさせて、\ruby{黙}{だま}りこんでしまいました。

\newpage
\thispagestyle{fancy}

\indent クねずみは\ruby{横}{よこ}の\ruby{方}{ほう}を\ruby{向}{む}いて、おひげをひっぱりながら、\ruby{横目}{よこ|め}でタねずみの\ruby{顔}{かお}を\ruby{見}{み}ていましたが、ずうっとしばらくたってから、あらんかぎり\ruby{声}{こえ}をひくくして、\\
「へい。そして。」と\ruby{言}{い}いました。ところがタねずみはもうすっかりこわくなって\ruby{物}{もの}が\ruby{言}{い}えませんでしたから、にわかに\ruby{一}{ひと}つていねいにおじぎをしました。そしてまるで\ruby{細}{ほそ}いかすれた\ruby{声}{こえ}で、\\
「さよなら。」と\ruby{言}{い}ってクねずみのおうちを\ruby{出}{で}て\ruby{行}{い}きました。\\
\indent クねずみは、そこであおむけにねころんで、\\
「ねずみ\ruby{競争}{きょう|そう}\ruby{新聞}{しん|ぶん}」を\ruby{手}{て}にとってひろげながら、\\
「ヘッ。タなどはなってないんだ。」とひとりごとを\ruby{言}{い}いました。\\
\indent さて、「ねずみ\ruby{競争}{きょう|そう}\ruby{新聞}{しん|ぶん}」というのは\ruby{実}{じつ}にいい\ruby{新聞}{しん|ぶん}です。これを\ruby{読}{よ}むと、ねずみ\ruby{仲間}{なか|ま}の\ruby{競争}{きょう|そう}のことはなんでもわかるのでした。ペねずみが、たくさんとうもろこしのつぶをぬすみためて、\ruby{大}{だい}\ruby{砂糖}{さ|とう}\ruby{持}{も}ちのパねずみと\ruby[<g>]{意地}{いじ}ばりの\ruby{競争}{きょう|そう}をしていることでも、ハねずみヒねずみフねずみの三\ruby{匹}{ひき}のむすめねずみが\ruby{学問}{がく|もん}の\ruby{競争}{きょう|そう}をやって、\ruby{比例}{ひ|れい}の\ruby{問題}{もん|だい}まで\ruby{来}{き}たとき、とうとう
三\ruby{匹}{びき}とも\ruby{頭}{あたま}がペチンと\ruby{裂}{さ}けたことでも、なんでもすっかり\ruby{出}{で}ているのでした。\\
\indent さあ、さあ、みなさん。\ruby{失礼}{しつ|れい}ですが、クねずみのきょうの\ruby{新聞}{しん|ぶん}を\ruby{読}{よ}むのを、お\ruby{聞}{き}きなさい。\\
「ええと、カマジン\ruby{国}{こく}の\ruby{飛行機}{ひ|こう|き}、プハラを\ruby{襲}{おそ}うと。なるほどえらいね。これはたいへんだ。まあしかし、ここまでは\ruby{来}{こ}ないから\ruby[<g>]{大丈夫}{だいじょうぶ}だ。ええと、ツェねずみの\ruby{行}{ゆ}くえ\ruby{不明}{ふ|めい}。ツェねずみというのはあの\ruby{意地}{い|じ}わるだな。こいつはおもしろい。\\
\indent \ruby{天井裏街}{てん|じょう|うら|がい}一\ruby{番地}{ばん|ち}、ツェ\ruby{氏}{し}は\ruby{昨夜}{さく|や}\ruby{行}{ゆ}くえ\ruby{不明}{ふ|めい}となりたり。\ruby{本社}{ほん|しゃ}のいちはやく\ruby{探知}{たん|ち}するところ

\newpage
\thispagestyle{fancy}
\fancyhead[R]
{
    \begin{spacing}{0.6}
        \fontsize{11}{20pt}\selectfont
        {
            P1 おわり\\
            1G:言\\
            2G:視\\
            3G:聴\\
            \indent\\
            \indent\\
            \indent\\
            \indent\\
            \indent\\
        }

    \end{spacing}
}
によればツェ\ruby{氏}{し}は\ruby{数日}{すう|じつ}\ruby{前}{まえ}より\textbf{はりがねせい}、
\textbf{ねずみとり}\ruby{氏}{し}と\ruby{交際}{こう|さい}を\ruby{結}{むす}びおりしが
\ruby[<g>]{一昨夜}{いっさくや}に\ruby{至}{いた}りて\ruby{両氏}{りょう|し}の\ruby{間}{あいだ}に\ruby[<m|]{多少}{た|しょう}\ruby[|m>]{感情}{かん|じょう}の\ruby{衝突}{しょう|とつ}ありたるもののごとし。
\ruby{台所街}{だい|どころ|がい}四\ruby{番地}{ばん|ち}ネ\ruby{氏}{し}の\ruby{談}{だん}によれば
\ruby[<g>]{昨夜}{さくや}もツェ\ruby{氏}{し}は、\textbf{はりがねせい}、\textbf{ねずみとり}\ruby{氏}{し}を
\ruby{訪問}{ほう|もん}したるがごとし、と。なお\ruby{床下}{ゆか|した}\ruby{通}{どお}り二十九\ruby{番地}{ばん|ち}
ポ\ruby{氏}{し}は、\ruby[<g>]{昨夜}{さくや}\ruby{深更}{しん|こう}より\ruby[<g>]{今朝}{けさ}にかけて、
ツェ\ruby{氏}{し}\ruby{並}{なら}びに\textbf{はりがねせい}、\textbf{ねずみとり}\ruby{氏}{し}の
\ruby{激}{はげ}しき\ruby{争論}{そう|ろん}、\ruby{時}{とき}に\ruby{格闘}{かく|とう}の\ruby{声}{こえ}を
\ruby{聞}{き}きたりと。\ruby{以上}{い|じょう}を\ruby{総合}{そう|ごう}するに、
\ruby{本}{ほん}\ruby{事件}{じ|けん}には、\textbf{はりがねせい}、\textbf{ねずみとり}\ruby{氏}{し}、
\ruby{最}{もっと}も\ruby{深}{ふか}き\ruby{関係}{かん|けい}を\ruby{有}{ゆう}するがごとし。
\ruby{本社}{ほん|しゃ}はさらに\ruby{深}{ふか}く\ruby{事件}{じ|けん}の\ruby{真相}{しん|そう}を\ruby{探知}{たん|ち}の
\ruby{上}{うえ}、\ruby{大}{おお}いに
\textbf{はりがねせい}、\textbf{ねずみとり}\ruby{氏}{し}に\ruby[<g>]{筆誅}{ひっちゅう}を
\ruby{加}{くわ}えんと\ruby{欲}{ほっ}す。と。
ははは、ふん、これはもう\ruby{疑}{うたが}いもない。ツェのやつめ、
ねずみとりに\ruby{食}{く}われたんだ。おもしろい。
そのつぎはと。なんだ、ええと、\ruby{新任}{しん|にん}ねずみ\ruby{会議}{かい|ぎ}\ruby{員}{いん}テ\ruby{氏}{し}。
エヘン、エヘン。エン。エッヘン。ヴェイヴェイ。なんだちくしょう。テなどがねずみ\ruby{会議}{かい|ぎ}\ruby{員}{いん}だなんて。
えい、おもしろくない。おれでもすればいいんだ。えい。おもしろくもない、\ruby[<g>]{散歩}{さんぽ}に\ruby{出}{で}よう。」
\begin{comment}
\end{comment}

%P2
\newpage
\thispagestyle{fancy}
\fancyhead[R]
{
    \begin{spacing}{0.6}
        \fontsize{11}{20pt}\selectfont
        {
            P2 ここから\\
            1G:聴\\
            2G:言\\
            3G:視\\
            \indent\\
            \indent\\
            \indent\\
            \indent\\
            \indent\\
        }

    \end{spacing}
}
\indent そこでクねずみは\ruby{散歩}{さん|ぽ}に\ruby{出}{で}ました。そしてプンプンおこりながら、
\ruby[<g>]{天井裏}{てんじょううら}\ruby{街}{がい}の\ruby{方}{ほう}へ\ruby{行}{い}く
\ruby[<g>]{途中}{とちゅう}で、二\ruby{匹}{ひき}のむかでが\ruby{親孝行}{おや|こう|こう}の
\ruby[<g>]{蜘蛛}{くも}の\ruby{話}{はなし}をしているのを\ruby{聞}{き}きました。\\
「ほんとうにね、そうはできないもんだよ。」\\
「ええ、ええ、\ruby{全}{まった}くですよ。それにあの\ruby{子}{こ}は、
\ruby[<g>]{自分}{じぶん}もどこかからだが\ruby{悪}{わる}いんですよ。
それだのにね、\ruby{朝}{あさ}は二\ruby{時}{じ}ごろから\ruby{起}{お}きて
\ruby{薬}{くすり}を\ruby{飲}{の}ませたり、おかゆをたいてやったり、\ruby{夜}{よる}だって
\ruby{寝}{ね}るのはいつもおそいでしょう。たいてい三\ruby{時}{じ}ごろでしょう。
ほんとうにからだがやすまるってないんでしょう。\ruby{感心}{かん|しん}ですねえ。」\\
「ほんとうにあんな\ruby{心}{こころ}がけのいい\ruby{子}{こ}は\ruby{今}{いま}ごろあり……。」\\
「エヘン、エヘン。」と、いきなりクねずみはどなって、おひげを\ruby{横}{よこ}の\ruby{方}{ほう}へ
ひっぱりました。\\
\indent むかではびっくりして、はなしもなにもそこそこに\ruby{別}{わか}れて\ruby{逃}{に}げて
\ruby{行}{い}ってしまいました。\\
\indent クねずみはそれからだんだん\ruby{天井裏}{てん|じょう|うら}\ruby{街}{がい}の\ruby{方}{ほう}へ
のぼって\ruby{行}{い}きました。\ruby{天井裏}{てん|じょう|うら}\ruby{街}{がい}のガランとした
\ruby{広}{ひろ}い\ruby{通}{とお}りでは、ねずみ\ruby{会議}{かい|ぎ}\ruby{員}{いん}のテねずみが
もう\ruby{一}{いっ}ぴきのね
ずみとはなしていました。\\
\indent クねずみはこわれたちり\ruby{取}{と}りのかげで\ruby{立}{たち}ちぎきをしておりました。\\
\indent テねずみが、\\
「それで、その、わたしの\ruby{考}{かんが}えではね、どうしてもこれは、その、
\ruby{共同}{きょう|どう}\ruby[<g>]{一致}{いっち}、\ruby{団結}{だん|けつ}、
\ruby{和睦}{わ|ぼく}の、セイシンで、やらんと、いかんね。」と\ruby{言}{い}いました。\\
\indent クねずみは、\\
「エヘン、エヘン。」と\ruby{聞}{き}こえないようにせきばらいをしました。\ruby{相手}{あい|て}のねずみは、
「へい。」
\newpage
\thispagestyle{fancy}
\fancyhead[R]
{
    \begin{spacing}{0.6}
        \fontsize{11}{20pt}\selectfont
        {
            P2\\
            1G:聴\\
            2G:言\\
            3G:視\\
            \indent\\
            \indent\\
            \indent\\
            \indent\\
            \indent\\
        }

    \end{spacing}
}
と\ruby{言}{い}って\ruby{考}{かんが}えているようです。\\
\indent テねずみははなしをつづけました。\\
「もしそうでないとすると、つまりその、\ruby{世界}{せ|かい}のシンポハッタツ、カイゼンカイリョウがそのつまりテイタイするね。」\\
「エン、エン、エイ、エイ。」クねずみはまたひくくせきばらいをしました。\\
\indent \ruby{相手}{あい|て}のねずみは、「へい。」と\ruby{言}{い}って\ruby{考}{かんが}えています。\\
「そこで、その、\ruby{世界}{せ|かい}\ruby{文明}{ぶん|めい}のシンポハッタツ、カイリョウカイゼンがテイタイすると、
\ruby{政治}{せい|じ}はもちろんケイザイ、ノウギョウ、ジツギョウ、コウギョウ、キョウイク、ビジュツそれからチョウコク、カイガ、それからブンガク、シバイ、ええと、エンゲキ、ゲイジュツ、ゴラク、そ
のほかタイイクなどが、ハッハッハ、たいへんそのどうもわるくなるね。」
テねずみはむつかしいことをあまりたくさん\ruby{言}{い}ったので、もう\ruby{愉快}{ゆ|かい}でたまらないようでした。
クねずみはそれがまたむやみにしゃくにさわって、「エン、エン。」と\ruby{聞}{き}こえないように、そしてできるだけ\ruby{高}{た
    か}くせきばらいをやって、にぎりこぶしをかためました。\\
\indent \ruby{相手}{あい|て}のねずみはやはり「へい。」と\ruby{言}{い}っております。\\
\indent テねずみはまたはじめました。\\
「そこでそのケイザイやゴラクが\ruby{悪}{わる}くなるというと、\ruby{不平}{ふ|へい}を\ruby{生}{しょう}じてブンレツを
\ruby{起}{お}こすというケッカにホウチャクするね。そうなるのは\ruby{実}{じつ}にそのわれわれのシンガイでフホンイであるから、
やはりその、ものごとは\ruby{共同}{きょう|どう}\ruby[<g>]{一致}{いっち}\ruby{団結}{だん|けつ}\ruby[<g>]{和睦}{わぼく}のセイシンでやらんといかんね。」

\newpage
\thispagestyle{fancy}

\indent クねずみはあんまりテねずみのことばが\ruby[<g>]{立派}{りっぱ}で、
\ruby{議論}{ぎ|ろん}がうまくできているのがしゃくにさわって、とうとうあらんかぎり、\\
「エヘン、エヘン。」とやってしまいました。するとテねずみはぶるるっとふるえて、
\ruby{目}{め}を\ruby{閉}{と}じて、\ruby{小}{ちい}さく\ruby{小}{ちい}さくちぢまりましたが、
だんだんそろりそろりと\ruby{延}{の}びて、そおっと\ruby{目}{め}をあいて、
それから\ruby{大声}{おお|ごえ}で\ruby{叫}{さけ}びました。\\
「こいつは、ブンレツだぞ。ブンレツ\ruby{者}{しゃ}だ。しばれ、しばれ。」と\ruby{叫}{さけ}びました。
すると\ruby{相手}{あい|て}のねずみは、まるでつぶてのようにクねずみに
\ruby{飛}{と}びかかってねずみの\ruby{捕}{と}り\ruby{繩}{つな}を\ruby{出}{だ}して、クルクルしばってしまいました。\\
\indent クねずみはくやしくてくやしくてなみだが\ruby{出}{で}ましたが、
どうしてもかないそうがありませんでしたから、しばらくじっとしておりました。
するとテねずみは\ruby{紙切}{かみ|き}れを\ruby{出}{だ}してするするするっと
\ruby{何}{なに}か\ruby{書}{か}いて\ruby{捕}{と}り\ruby{手}{て}のねずみに\ruby{渡}{わた}し
ました。\\
\indent \ruby{捕}{と}り\ruby{手}{て}のねずみは、しばられてごろごろころがっているクねずみの
\ruby{前}{まえ}に\ruby{来}{き}て、すてきにおごそかな\ruby{声}{こえ}でそれを\ruby{読}{よ}みはじめました。\\
「クねずみはブンレツ\ruby{者}{しゃ}によりて、みんなの\ruby{前}{まえ}にて\ruby{暗殺}{あん|さつ}すべし。」
クねずみは\ruby{声}{こえ}をあげてチュウチュウ\ruby{泣}{な}きました。\\
「さあ、ブンレツ\ruby{者}{しゃ}。あるけ、\ruby{早}{はや}く。」と、\ruby{捕}{と}り\ruby{手}{て}のねずみは\ruby{言}{い}いました。さあ、そこでクねずみはすっかり\ruby{恐}{おそ}れ\ruby{入}{い}ってしおしおと\ruby{立}{た}ちあがりました。あっちからもこっちからもねずみがみんな\ruby{集}{あつ}まって\ruby{来}{き}て、

\newpage
\thispagestyle{fancy}
\fancyhead[R]
{
    \begin{spacing}{0.6}
        \fontsize{11}{20pt}\selectfont
        {
            P2 おわり\\
            1G:聴\\
            2G:言\\
            3G:視\\
            \indent\\
            \indent\\
            \indent\\
            \indent\\
            \indent\\
        }

    \end{spacing}
}
「どうもいい\ruby{気味}{き|み}だね。いつでもエヘンエヘンと\ruby{言}{い}ってばかりいたやつなんだ。」\\
「やっぱり\ruby{分裂}{ぶん|れつ}していたんだ。」\\
「あいつが\ruby{死}{し}んだらほんとうにせいせいするだろうね。」
というような\ruby{声}{こえ}ばかりです。\\
\indent \ruby{捕}{と}り\ruby{手}{て}のねずみは、いよいよ\ruby{白}{しろ}いたすきをかけて、
\ruby{暗殺}{あん|さつ}のしたくをはじめました。\\
\indent その\ruby{時}{とき}みんなのうしろの\ruby{方}{ほう}で、
フウフウと\ruby{言}{い}うひどい\ruby{音}{おと}が\ruby{聞}{き}こえ、
\ruby{二}{ふた}つの\ruby{目玉}{め|だま}が\ruby{火}{ひ}のように\ruby{光}{ひか}って
\ruby{来}{き}ました。それは\ruby{例}{れい}の\ruby{猫}{ねこ}\ruby{大将}{たい|しょう}でした。\\
「ワーッ。」とねずみはみんなちりぢり\ruby{四方}{し|ほう}に\ruby{逃}{に}げました。\\
「\ruby{逃}{に}がさんぞ。コラッ。」と\ruby{猫}{ねこ}\ruby{大将}{たい|しょう}は
その一\ruby{匹}{ひき}を\ruby{追}{お}いかけましたが、もうせまいすきまへずうっと\ruby{深}{ふか}くも
ぐり\ruby{込}{こ}んでしまったので、いくら\ruby{猫}{ねこ}\ruby{大将}{たい|しょう}が
\ruby{手}{て}をのばしてもとどきませんでした。

\newpage
\thispagestyle{fancy}
\fancyhead[R]
{
    \begin{spacing}{0.6}
        \fontsize{11}{20pt}\selectfont
        {
            P3 ここから\\
            1G:視\\
            2G:聴\\
            3G:言\\
            \indent\\
            \indent\\
            \indent\\
            \indent\\
            \indent\\
        }

    \end{spacing}
}

\noindent \ruby{猫}{ねこ}\ruby{大将}{たい|しょう}は「チェッ。」と\ruby{舌打}{し|たう}ちをして
\ruby{戻}{もど}って\ruby{来}{き}ましたが、
クねずみのただ一\ruby{匹}{ひき}しばられて\ruby{残}{のこ}っているのを\ruby{見}{み}て、
びっくりして\ruby{言}{い}いました。\\
「\ruby{貴様}{き|さま}はなんと\ruby{言}{い}うものだ。」
クねずみはもう\ruby{落}{お}ち\ruby{着}{つ}いて\ruby{答}{こた}えました。\\
「クと\ruby{申}{もう}します。」\\
「フ、フ、そうか、なぜこんなにしているんだ。」\\
「\ruby{暗殺}{あん|さつ}されるためです。」\\
「フ、フ、フ。そうか。それはかあいそうだ。よしよし、おれが\ruby{引}{ひ}き\ruby{受}{う}けてやろう。
おれのうちへ\ruby{来}{こ}い。ちょうどおれのうちでは、\ruby{子供}{こ|ども}が四\ruby{人}{にん}できて、それに\ruby{家庭教師}{か|てい|きょう|し}がなくて\ruby{困}{こま}っているところなんだ。\ruby{来}{こ}い。」\\
\indent \ruby{猫}{ねこ}\ruby{大将}{たい|しょう}はのそのそ\ruby{歩}{ある}きだしました。\\
\indent クねずみはこわごわあとについて\ruby{行}{い}きました。\ruby{猫}{ねこ}のおうちはどうもそれは\ruby[<g>]{立派}{りっぱ}なもんでした。\ruby{紫色}{むらさき|いろ}の\ruby{竹}{たけ}で\ruby{編}{あ}んであって\ruby{中}{なか}はわらや\ruby{布}{ぬの}きれでホクホクしていました。おまけにちゃあんとご\ruby{飯}{はん}を\ruby{入}{い}れる\ruby{道具}{どう|ぐ}さえあったのです。\\
\indent そしてその\ruby{中}{なか}に、\ruby{猫}{ねこ}\ruby{大将}{たい|しょう}の\ruby{子供}{こ|ども}が四\ruby{人}{にん}、やっと\ruby{目}{め}をあいて、にゃあにゃあと\ruby{鳴}{な}いておりました。\\
\indent \ruby{猫}{ねこ}\ruby{大将}{たい|しょう}は\ruby{子供}{こ|ども}らを\ruby{一}{ひと}つずつなめてやってから\ruby{言}{い}いました。\\
「お\ruby{前}{まえ}たちはもう\ruby{学問}{がく|もん}をしないといけない。ここへ\ruby{先生}{せん|せい}をたのんで\ruby{来}{き}たからな。よく\ruby{習}{なら}うんだよ。\ruby{決}{けっ}して\ruby{先生}{せん|せい}を\ruby{食}{た}べてしまったりしてはいかんぞ。」

\newpage
\thispagestyle{fancy}
\fancyhead[R]
{
    \begin{spacing}{0.6}
        \fontsize{11}{20pt}\selectfont
        {
            P3\\
            1G:視\\
            2G:聴\\
            3G:言\\
            \indent\\
            \indent\\
            \indent\\
            \indent\\
            \indent\\
        }

    \end{spacing}
}

\indent \ruby{子供}{こ|ども}らはよろこんでニヤニヤ\ruby{笑}{わら}って\ruby[<g>]{口々}{くちぐち}に、\\
「おとうさん、ありがとう。きっと\ruby{習}{なら}うよ。\ruby{先生}{せん|せい}を\ruby{食}{た}べてしまったりしないよ。」と\ruby{言}{い}いました。\\
\indent クねずみはどうも\ruby{思}{おも}わず\ruby{足}{あし}がブルブルしました。\\
\indent \ruby{猫}{ねこ}\ruby{大将}{たい|しょう}が\ruby{言}{い}いました。\\
「\ruby{教}{おし}えてやってくれ。おもに\ruby{算術}{さん|じゅつ}をな。」\\
「へい。しょう、しょう、\ruby{承知}{しょう|ち}いたしました。」とクねずみが\ruby{答}{こた}えました。\\
\indent \ruby{猫}{ねこ}\ruby{大将}{たい|しょう}はきげんよくニャーと\ruby{鳴}{な}いてするりと\ruby{向}{む}こうへ\ruby{行}{い}ってしまいました。\\
\indent \ruby{子供}{こ|ども}らが\ruby{叫}{さけ}びました。\\
「\ruby{先生}{せん|せい}、\ruby{早}{はや}く\ruby{算術}{さん|じゅつ}を\ruby{教}{おし}えてください。\ruby{先生}{せん|せい}。\ruby{早}{はや}く。」\\
\indent クねずみはさあ、これはいよいよ\ruby{教}{おし}えないといかんと\ruby{思}{おも}いましたので、\ruby{口早}{くち|ばや}に\ruby{言}{い}いました。\\
「一に一をたすと二です。」\\
「そうだよ。」\ruby{子供}{こ|ども}らが\ruby{言}{い}いました。\\
「一から一を\ruby{引}{ひ}くとなんにもなくなります。」\\
「わかったよ。」\\
\indent \ruby{子供}{こ|ども}らが\ruby{叫}{さけ}びました。\\
「一に一をかけると一です。」\\
「きまってるよ。」と\ruby{猫}{ねこ}の\ruby{子供}{こ|ども}らが\ruby{目}{め}をりんと\ruby{張}{は}ったまま\ruby{答}{こた}えました。\\
「一を一で\ruby{割}{わ}ると一です。」

\newpage
\thispagestyle{fancy}

「それでいいよ。」と\ruby{猫}{ねこ}の\ruby{子供}{こ|ども}らがよろこんで\ruby{叫}{さけ}びました。そこでクねずみはすっかりのぼせてしまいました。\\
「一に二をたすと三です。」\\
「\ruby{合}{あ}ってるよ。」\\
「一から二を\ruby{引}{ひ}くと……」と\ruby{言}{い}おうとしてクねずみは、はっとつまってしまいました。\\
\indent すると\ruby{猫}{ねこ}の\ruby{子供}{こ|ども}らは\ruby{一度}{いち|ど}に\ruby{叫}{さけ}びました。\\
「一から二は\ruby{引}{ひ}かれないよ。」\\
\indent クねずみはあんまり\ruby{猫}{ねこ}の\ruby{子供}{こ|ども}らがかしこいので、すっかりむしゃくしゃして、また\ruby{早口}{はや|くち}に\ruby{言}{い}いました。そうでしょう。クねずみは\\
いちばんはじめの一に一をたして二をおぼえるのに\ruby{半年}{はん|とし}かかったのです。\\
「一に二をかけると二です。」\\
「そうともさ。」\\
「一を二で\ruby{割}{わ}ると……。」クねずみはまたつまってしまいました。すると\ruby{猫}{ねこ}の\ruby{子供}{こ|ども}らはまた\ruby{一度}{いち|ど}に\ruby{声}{こえ}をそろえて、\\
「一\ruby{割}{わ}る二では\ruby{半分}{はん|ぶん}だよ。」と\ruby{叫}{さけ}びました。\\
\indent クねずみはあんまり\ruby{猫}{ねこ}の\ruby{子供}{こ|ども}らの\ruby{賢}{かしこ}いのがしゃくにさわって、\ruby{思}{おも}わず\\
「エヘン。エヘン。エイ。エイ。」とやりました。\\
すると\ruby{猫}{ねこ}の\ruby{子供}{こ|ども}らは、しばらくびっくりしたように、\ruby{顔}{かお}を\ruby{見合}{み|あ}わせていましたが、
\newpage
\thispagestyle{fancy}
\fancyhead[R]
{
    \begin{spacing}{0.6}
        \fontsize{11}{20pt}\selectfont
        {
            P3 おわり\\
            1G:視\\
            2G:聴\\
            3G:言\\
            \indent\\
            \indent\\
            \indent\\
            \indent\\
            \indent\\
        }

    \end{spacing}
}
やがてみんな\ruby{一度}{いち|ど}に\ruby{立}{た}ちあがって、\\
「なんだい。ねずめ、\ruby{人}{ひと}をそねみやがったな。」と\ruby{言}{い}いながらクねずみの\ruby{足}{あし}を\ruby{一}{いっ}ぴきが\ruby{一}{ひと}つずつかじりました。\\
\indent クねずみは\ruby{非常}{ひ|じょう}にあわててばたばたして、\ruby{急}{いそ}いで「エヘン、エヘン、エイ、エイ。」とやりましたがもういけませんでした。\\
\indent クねずみはだんだん\ruby{四方}{し|ほう}の\ruby{足}{あし}から\ruby{食}{く}われて\ruby{行}{い}って、とうとうおしまいに四ひきの\ruby{子猫}{こ|ねこ}は、クねずみの\ruby{胃}{い}の\ruby{腑}{ふ}のところで\ruby{頭}{あたま}をコツンとぶっつけました。\\
\indent そこへ\ruby{猫}{ねこ}\ruby{大将}{たい|しょう}が\ruby{帰}{かえ}って\ruby{来}{き}て、\\
「\ruby{何}{なに}か\ruby{習}{なら}ったか。」とききました。\\
「ねずみをとることです。」と四ひきがいっしょに\ruby{答}{こた}えました。

\newpage
\nolinenumbers
底本:「童話集 銀河鉄道の夜 他十四編」谷川徹三編、岩波文庫、岩波書店
\\
\indent 1951(昭和26)年10月25日第1刷発行\\
\indent 1966(昭和41)年7月16日第18刷改版発行\\
\indent 2000(平成12)年5月25日第71刷発行\\
底本の親本:「宮沢賢治全集 第八巻」筑摩書房\\
\indent 1956(昭和31)年10月\\
入力:のぶ\\
校正:鈴木厚司\\
2003年8月3日作成\\
2008年2月29日修正\\
青空文庫作成ファイル:\\
このファイルは、インターネットの図書館、"http://www.aozora.gr.jp/">青空文庫(http://www.aozora.gr.jp/)で作られました。入力、校正、制作にあたったのは、ボランティアの皆さんです。\\
\\

\end{document}