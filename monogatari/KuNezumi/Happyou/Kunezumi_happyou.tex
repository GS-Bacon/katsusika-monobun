\documentclass[
a4paper,
10pt,
book]
{tarticle}
% A4 20pt book tarticle→縦書き
\usepackage{pxrubrica} %ルビ追加
\usepackage[deluxe]{otf}
\usepackage{comment} %ブロックでコメントアウト
\usepackage{fancyhdr}
\usepackage{lineno} %文章に行番号を振る
\usepackage{plext}
\usepackage{graphics}
\usepackage{xcolor}% 文字に色を付ける
\usepackage{pdfpages}%面付設定

\renewcommand{\headrulewidth}{1pt} %ヘッダー罫線を消す
\renewcommand\linenumberfont{\normalfont\bfseries\large}
\renewcommand\LineNumber{{\color{gray}\raisebox{-2.5mm}{\rotatebox{90}{\thelinenumber}}}}%行番号を回転とズレた分調整

\AtBeginDvi{\special{papersize=210mm,297mm}} %A4サイズ 210x297
\setlength{\topmargin}{15mm} % ページ上部余白の設定
\addtolength{\topmargin}{-1in} % 初期設定の1インチ分を引いておく。

\setlength{\oddsidemargin}{22mm} % 同、奇数ページ左。
\addtolength{\oddsidemargin}{-1in}

\setlength{\evensidemargin}{28mm} % 同、偶数ページ左。
\addtolength{\evensidemargin}{-1in}

\setlength{\textwidth}{245mm} % 文書領域の幅(上下)。縦書と横書でパラメータ(width / height)の向きが変わる。

\setlength{\textheight}{160mm} % 文書領域の幅(左右)。
\setlength{\headsep}{10mm} % ヘッダ下端と本文領域との間隔。私の場合、ヘッダをつけぬので、この辺の数値は0である。しかし headsep って何や。headseparation? headseparator?
\setlength{\headheight}{5mm} % headheight → ヘッダの高さ。ヘッダはつけないが、設定しておかぬと TeX の自動配置の際にデフォルト値(知らんけど)を入れられそう。
\setlength{\topskip}{0mm} % ここは改善の余地あり→下段 [参考] "行位置が揃った段組" を見ること。
\setlength{\parskip}{0pt} % 段落間の自動空白調整を切る。parskip って何の略や。parindent っちゅうのもあるけど、par って何なんや? → par とは paragraph(段落)のことらしい。 parindent は、1行目の字下げ幅。

\fancypagestyle{plain}{\pagestyle{fancy}}
%1ページあたりの行数、文字数を設定するためのなんかよくわからんやつ
\makeatletter
\def\mojiparline#1{
    \newcounter{mpl}
    \setcounter{mpl}{#1}
    \@tempdima=\linewidth
    \advance\@tempdima by-\value{mpl}zw
    \addtocounter{mpl}{-1}
    \divide\@tempdima by \value{mpl}
    \advance\kanjiskip by\@tempdima
    \advance\parindent by\@tempdima
    }
    \makeatother
    \def\linesparpage#1{
        \baselineskip=\textheight
        \divide\baselineskip by #1
   }
\begin{document}

\Large
%normalsizeを10ptに設定して14pt

% 一行あたり文字数の指定
\mojiparline{42}
% 1ページあたり行数の指定
\linesparpage{18}

%\fontsize{16pt}{28pt}\selectfont
\rubysetup{<m>} %ルビのデフォルト設定 モノルビで
\mcfamily %フォント設定 ヒラノギ細字を使う

%表紙
\thispagestyle{empty}
{\fontsize{30pt}{24pt}\selectfont
クねずみ\\
\begin{flushright}
    \ruby{宮沢}{みや|ざわ}\ruby{賢治}{けん|じ}\\
\end{flushright}
}
\newpage
\setcounter{page}{1}
\pagewiselinenumbers
\thispagestyle{fancy}\fancyhead[R]{ここから P1}{
\indent クという\ruby{名前}{な|まえ}のねずみがありました。たいへん\ruby{高慢}{ごう|まん}でそれにそねみ\ruby{深}{ぶか}くって、
\ruby{自分}{じ|ぶん}をねずみの\ruby{仲間}{なか|ま}の一\ruby{番}{ばん}の\ruby{学者}{がく|しゃ}と\ruby{思}{おも}っていました。
ほかのねずみが\ruby{何}{なに}か\ruby[<g>]{生意気}{なまいき}なこと
を\ruby{言}{い}うとエヘンエヘンと\ruby{言}{い}うのが\ruby{癖}{くせ}でした。\\
\indent クねずみのうちへ、ある\ruby{日}{ひ}、\ruby{友}{とも}だちのタねずみがやって\ruby{来}{き}ました。\\
\indent さてタねずみはクねずみに\ruby{言}{い}いました。\\
「\ruby[<g>]{今日}{こんにち}は、クさん。いいお\ruby{天気}{て|んき}です。」\\
「いいお\ruby{天気}{てん|き}です。\ruby{何}{なに}かいいものを\ruby{見}{み}つけましたか。」\\
「いいえ。どうも\ruby{不景気}{ふ|けい|き}ですね。どうでしょう。これからの\ruby{景気}{けい|き}は。」\\
「さあ、あなたはどう\ruby{思}{おも}いますか。」\\
「そうですね。しかしだんだんよくなるのじゃないでしょうか。オウベイのキンユウはしだいにヒッパクをテイしたそう……。」\\
「エヘン、エヘン。」いきなりクねずみが\ruby{大}{おお}きなせきばらいをしましたので、タねずみはびっくりして\ruby{飛}{と}びあがりました。
クねずみは\ruby{横}{よこ}を\ruby{向}{む}いたまま、ひげを\ruby{一}{ひと}つぴんとひねって、それから\ruby{口}{くち}の\ruby{中}{なか}で、\\
「ヘイ、それから。」と\ruby{言}{い}いました。\\
\indent タねずみはやっと\ruby{安心}{あん|しん}してまたおひざに\ruby{手}{て}を\ruby{置}{お}いてすわりました。\\
\indent クねずみもやっとまっすぐを\ruby{向}{む}いて\ruby{言}{い}いました。\\
「\ruby{先}{せん}ころの\ruby{地震}{じ|しん}にはおどろきましたね。」

}
\newpage
\thispagestyle{fancy}
\fancyhead[R]{\empty}
\fancyhead[C]{P1}
「\ruby{全}{まった}くです。」\\
「あんな\ruby{大}{おお}きいのは\ruby{私}{わたし}もはじめてですよ。」\\
「ええ、ジョウカドウでしたねえ。シンゲンはなんでもトウケイ四十二\ruby{度}{ど}二\ruby{分}{ふん}ナンイ……。」\\
「エヘン、エヘン。」\\
\indent クねずみはまたどなりました。\\
\indent タねずみはまた\ruby{面}{めん}くらいましたが、さっきほどではありませんでした。\\
\indent クねずみはやっと\ruby{気}{き}を\ruby{直}{なお}して\ruby{言}{い}いました。\\
「\ruby{天気}{てん|き}もよくなりましたね。あなたは\ruby{何}{なに}かうまい\ruby[<g>]{仕掛}{しか}けをしておきましたか。」\\
「いいえ、なんにもしておきません。しかし、\ruby[<g>]{今度}{こんど} \ruby{天気}{てん|き}が\ruby{長}{なが}くつづいたら、\ruby{私}{わたし}は\ruby{少}{すこ}し\ruby{畑}{はたけ}の\ruby{方}{ほう}へ\ruby{出}{で}てみようと\ruby{思}{おも}うんです。」\\
「\ruby{畑}{はたけ}には\ruby{何}{なに}かいいことがありますか。」\\
「\ruby{秋}{あき}ですからとにかく\ruby{何}{なに}かこぼれているだろうと\ruby{思}{おも}います。\ruby{天気}{てん|き}さえよければいいのですがね。」\\
「どうでしょう。\ruby{天気}{てん|き}はいいでしょうか。」\\
「そうですね、\ruby{新聞}{しん|ぶん}に\ruby{出}{で}ていましたが、オキナワレットウにハッセイしたテイキアツは\ruby[<g>]{次第}{しだい}にホクホクセイのほうへシンコウ……。」\\
「エヘン、エヘン。」クねずみはまたいやなせきばらいをやりましたので、タねずみはこんどというこんどはすっかりびっくりして\ruby{半分}{はん|ぶん}\ruby{立}{た}ちあがって、ぶるぶるふるえて\ruby{目}{め}をパチパチさせて、\ruby{黙}{だま}りこんでしまいました。

\newpage
\thispagestyle{fancy}
\fancyhead[R]{\empty}
\fancyhead[C]{P1}
\indent クねずみは\ruby{横}{よこ}の\ruby{方}{ほう}を\ruby{向}{む}いて、おひげをひっぱりながら、\ruby{横目}{よこ|め}でタねずみの\ruby{顔}{かお}を\ruby{見}{み}ていましたが、ずうっとしばらくたってから、あらんかぎり\ruby{声}{こえ}をひくくして、\\
「へい。そして。」と\ruby{言}{い}いました。ところがタねずみはもうすっかりこわくなって\ruby{物}{もの}が\ruby{言}{い}えませんでしたから、にわかに\ruby{一}{ひと}つていねいにおじぎをしました。そしてまるで\ruby{細}{ほそ}いかすれた\ruby{声}{こえ}で、\\
「さよなら。」と\ruby{言}{い}ってクねずみのおうちを\ruby{出}{で}て\ruby{行}{い}きました。\\
\indent クねずみは、そこであおむけにねころんで、\\
「ねずみ\ruby{競争}{きょう|そう}\ruby{新聞}{しん|ぶん}」を\ruby{手}{て}にとってひろげながら、\\
「ヘッ。タなどはなってないんだ。」とひとりごとを\ruby{言}{い}いました。\\
\indent さて、「ねずみ\ruby{競争}{きょう|そう}\ruby{新聞}{しん|ぶん}」というのは\ruby{実}{じつ}にいい\ruby{新聞}{しん|ぶん}です。これを\ruby{読}{よ}むと、ねずみ\ruby{仲間}{なか|ま}の\ruby{競争}{きょう|そう}のことはなんでもわかるのでした。ペねずみが、たくさんとうもろこしのつぶをぬすみためて、\ruby{大}{だい}\ruby{砂糖}{さ|とう}\ruby{持}{も}ちのパねずみと\ruby[<g>]{意地}{いじ}ばりの\ruby{競争}{きょう|そう}をしていることでも、ハねずみヒねずみフねずみの三\ruby{匹}{ひき}のむすめねずみが\ruby{学問}{がく|もん}の\ruby{競争}{きょう|そう}をやって、\ruby{比例}{ひ|れい}の\ruby{問題}{もん|だい}まで\ruby{来}{き}たとき、とうとう
三\ruby{匹}{びき}とも\ruby{頭}{あたま}がペチンと\ruby{裂}{さ}けたことでも、なんでもすっかり\ruby{出}{で}ているのでした。\\
\indent さあ、さあ、みなさん。\ruby{失礼}{しつ|れい}ですが、クねずみのきょうの\ruby{新聞}{しん|ぶん}を\ruby{読}{よ}むのを、お\ruby{聞}{き}きなさい。\\
「ええと、カマジン\ruby{国}{くに}の\ruby{飛行機}{ひ|こう|き}、プハラを\ruby{襲}{おそ}うと。なるほどえらいね。これはたいへんだ。まあしかし、ここまでは\ruby{来}{こ}ないから\ruby[<g>]{大丈夫}{だいじょうぶ}だ。ええと、ツェねずみの\ruby{行}{ゆ}くえ\ruby{不明}{ふ|めい}。ツェねずみというのはあの\ruby{意地}{い|じ}わるだな。こいつはおもしろい。\\
\indent \ruby[<g>]{天井裏}{てんじょううら}\ruby{街}{まち}一\ruby{番地}{ばん|ち}、ツェ\ruby{氏}{し}は\ruby{昨夜}{さく|や}\ruby{行}{ゆ}くえ\ruby{不明}{ふ|めい}となりたり。\ruby{本社}{ほん|しゃ}のいちはやく\ruby{探知}{たん|ち}するところ

\newpage
\thispagestyle{fancy}
\fancyhead[R]{\empty}
\fancyhead[C]{P1おわり}
によればツェ\ruby{氏}{し}は\ruby{数日}{すう|じつ}\ruby{前}{まえ}より\textbf{はりがねせい}、
\textbf{ねずみとり}\ruby{氏}{し}と\ruby{交際}{こう|さい}を\ruby{結}{むす}びおりしが
\ruby[<g>]{一昨夜}{いっさくや}に\ruby{至}{いた}りて\ruby{両氏}{りょう|し}の\ruby{間}{あいだ}に\ruby{多少}{た|しょう}
\ruby{感情}{かん|じょう}の\ruby{衝突}{しょう|とつ}ありたるもののごとし。
\ruby{台所}{だい|どころ}\ruby{街}{まち}四\ruby{番地}{ばん|ち}ネ\ruby{氏}{し}の\ruby{談}{だん}によれば
\ruby[<g>]{昨夜}{さくや}もツェ\ruby{氏}{し}は、\textbf{はりがねせい}、\textbf{ねずみとり}\ruby{氏}{し}を
\ruby{訪問}{ほう|もん}したるがごとし、と。なお\ruby{床下}{ゆか|した}\ruby{通}{どお}り二十九\ruby{番地}{ばん|ち}
ポ\ruby{氏}{し}は、\ruby[<g>]{昨夜}{さくや}\ruby{深更}{しん|こう}より\ruby[<g>]{今朝}{けさ}にかけて、
ツェ\ruby{氏}{し}\ruby{並}{なら}びに\textbf{はりがねせい}、\textbf{ねずみとり}\ruby{氏}{し}の
\ruby{激}{はげ}しき\ruby{争論}{そう|ろん}、\ruby{時}{とき}に\ruby{格闘}{かく|とう}の\ruby{声}{こえ}を
\ruby{聞}{き}きたりと。\ruby{以上}{い|じょう}を\ruby{総合}{そう|ごう}するに、
\ruby{本}{ほん}\ruby{事件}{じ|けん}には、\textbf{はりがねせい}、\textbf{ねずみとり}\ruby{氏}{し}、
\ruby{最}{もっと}も\ruby{深}{ふか}き\ruby{関係}{かん|けい}を\ruby{有}{ゆう}するがごとし。
\ruby{本社}{ほん|しゃ}はさらに\ruby{深}{ふか}く\ruby{事件}{じ|けん}の\ruby{真相}{しん|そう}を\ruby{探知}{たん|ち}の
\ruby{上}{うえ}、\ruby{大}{おお}いに
\textbf{はりがねせい}、\textbf{ねずみとり}\ruby{氏}{し}に\ruby[<g>]{筆誅}{ひっちゅう}を
\ruby{加}{くわ}えんと\ruby{欲}{ほっ}す。と。
ははは、ふん、これはもう\ruby{疑}{うたが}いもない。ツェのやつめ、
ねずみとりに\ruby{食}{く}われたんだ。おもしろい。
そのつぎはと。なんだ、ええと、\ruby{新任}{しん|にん}ねずみ\ruby{会議}{かい|ぎ}\ruby{員}{いん}テ\ruby{氏}{し}。
エヘン、エヘン。エン。エッヘン。ヴェイヴェイ。なんだちくしょう。テなどがねずみ\ruby{会議}{かい|ぎ}\ruby{員}{いん}だなんて。
えい、おもしろくない。おれでもすればいいんだ。えい。おもしろくもない、\ruby[<g>]{散歩}{さんぽ}に\ruby{出}{で}よう。」
\begin{comment}
\end{comment}

%P2
\newpage
\thispagestyle{fancy}
\fancyhead[C]{\empty}
\fancyhead[R]{ここから P2}
\indent そこでクねずみは\ruby{散歩}{さん|ぽ}に\ruby{出}{で}ました。そしてプンプンおこりながら、
\ruby[<g>]{天井裏}{てんじょううら}\ruby{街}{まち}の\ruby{方}{ほう}へ\ruby{行}{い}く
\ruby[<g>]{途中}{とちゅう}で、二\ruby{匹}{ひき}のむかでが\ruby{親孝行}{おや|こう|こう}の
\ruby[<g>]{蜘蛛}{くも}の\ruby{話}{はなし}をしているのを\ruby{聞}{き}きました。\\
「ほんとうにね、そうはできないもんだよ。」\\
「ええ、ええ、\ruby{全}{まった}くですよ。それにあの\ruby{子}{こ}は、
\ruby[<g>]{自分}{じぶん}もどこかからだが\ruby{悪}{わる}いんですよ。
それだのにね、\ruby{朝}{あさ}は二\ruby{時}{じ}ごろから\ruby{起}{お}きて
\ruby{薬}{くすり}を\ruby{飲}{の}ませたり、おかゆをたいてやったり、\ruby{夜}{よる}だって
\ruby{寝}{ね}るのはいつもおそいでしょう。たいてい三\ruby{時}{じ}ごろでしょう。
ほんとうにからだがやすまるってないんでしょう。\ruby{感心}{かん|しん}ですねえ。」\\
「ほんとうにあんな\ruby{心}{こころ}がけのいい\ruby{子}{こ}は\ruby{今}{いま}ごろあり……。」\\
「エヘン、エヘン。」と、いきなりクねずみはどなって、おひげを\ruby{横}{よこ}の\ruby{方}{ほう}へ
ひっぱりました。\\
\indent むかではびっくりして、はなしもなにもそこそこに\ruby{別}{わか}れて\ruby{逃}{に}げて
\ruby{行}{い}ってしまいました。\\
\indent クねずみはそれからだんだん\ruby{天井裏}{てん|じょう|うら}\ruby{街}{まち}の\ruby{方}{ほう}へ
のぼって\ruby{行}{い}きました。\ruby{天井裏}{てん|じょう|うら}\ruby{街}{まち}のガランとした
\ruby{広}{ひろ}い\ruby{通}{とお}りでは、ねずみ\ruby{会議}{かい|ぎ}\ruby{員}{いん}のテねずみが
もう\ruby{一}{いっ}ぴきのね
ずみとはなしていました。\\
\indent クねずみはこわれたちり\ruby{取}{と}りのかげで\ruby{立}{たち}ちぎきをしておりました。\\
\indent テねずみが、\\
「それで、その、わたしの\ruby{考}{かんが}えではね、どうしてもこれは、その、
\ruby{共同}{きょう|どう}\ruby[<g>]{一致}{いっち}、\ruby{団結}{だん|けつ}、
\ruby{和睦}{わ|ぼく}の、セイシンで、やらんと、いかんね。」と\ruby{言}{い}いました。\\
\indent クねずみは、\\
「エヘン、エヘン。」と\ruby{聞}{き}こえないようにせきばらいをしました。\ruby{相手}{あい|て}のねずみは、
「へい。」
\newpage
\thispagestyle{fancy}
\fancyhead[R]{\empty}
\fancyhead[C]{P2}
と言って考えているようです。\\
\indent テねずみははなしをつづけました。\\
「もしそうでないとすると、つまりその、世界のシンポハッタツ、カイゼンカイリョウがそのつまりテイタイするね。」\\
「エン、エン、エイ、エイ。」クねずみはまたひくくせきばらいをしました。\\
\indent 相手のねずみは、「へい。」と言って考えています。\\
「そこで、その、世界文明のシンポハッタツ、カイリョウカイゼンがテイタイすると、政治はもちろんケイザイ、
ノウギョウ、ジツギョウ、コウギョウ、キョウイク、ビジュツそれからチョウコク、カイガ、それからブンガク、シバイ、
ええと、エンゲキ、ゲイジュツ、ゴラク、そのほかタイイクなどが、ハッハッハ、たいへんそのどうもわるくなるね。」
テねずみはむつかしいことをあまりたくさん言ったので、もう愉快でたまらないようでした。クねずみはそれがまたむやみにしゃくにさわって、
「エン、エン。」と聞こえないように、そしてできるだけ高くせきばらいをやって、にぎりこぶしをかためました。\\
\indent 相手のねずみはやはり「へい。」と言っております。\\
\indent テねずみはまたはじめました。\\
「そこでそのケイザイやゴラクが悪くなるというと、不平を生じてブンレツを起こすというケッカにホウチャクするね。
そうなるのは実にそのわれわれのシンガイでフホンイであるから、やはりその、ものごとは共同一致団結和睦のセイシンでやらんといかんね。」
\newpage
\thispagestyle{fancy}
\fancyhead[R]{\empty}
\fancyhead[C]{P2}
\indent クねずみはあんまりテねずみのことばが立派で、議論がうまくできているのがしゃくにさわって、とうとうあらんかぎり、\\
「エヘン、エヘン。」とやってしまいました。するとテねずみはぶるるっとふるえて、目を閉じて、小さく小さくちぢまりましたが、
だんだんそろりそろりと延びて、そおっと目をあいて、それから大声で叫びました。\\
「こいつは、ブンレツだぞ。ブンレツ者だ。しばれ、しばれ。」と叫びました。すると相手のねずみは、
まるでつぶてのようにクねずみに飛びかかってねずみの\ruby{捕}{と}り\ruby{繩}{なわ}を出して、クルクルしばってしまいました。\\
\indent クねずみはくやしくてくやしくてなみだが出ましたが、どうしてもかないそうがありませんでしたから、
しばらくじっとしておりました。するとテねずみは紙切れを出してするするするっと何か書いて捕り手のねずみに渡しました。\\
\indent 捕り手のねずみは、しばられてごろごろころがっているクねずみの前に来て、すてきにおごそかな声でそれを読みはじめました。\\
「クねずみはブンレツ者によりて、みんなの前にて暗殺すべし。」クねずみは声をあげてチュウチュウ泣きました。\\
「さあ、ブンレツ者。あるけ、早く。」と、捕り手のねずみは言いました。
さあ、そこでクねずみはすっかり恐れ入ってしおしおと立ちあがりました。あっちからもこっちからもねずみがみんな集まって来て、

\newpage
\thispagestyle{fancy}
\fancyhead[R]{\empty}
\fancyhead[C]{P2 おわり}
「どうもいい気味だね。いつでもエヘンエヘンと言ってばかりいたやつなんだ。」\\
「やっぱり分裂していたんだ。」\\
「あいつが死んだらほんとうにせいせいするだろうね。」というような声ばかりです。\\
\indent 捕り手のねずみは、いよいよ白いたすきをかけて、暗殺のしたくをはじめました。\\
\indent その時みんなのうしろの方で、フウフウと言うひどい音が聞こえ、二つの目玉が火のように光って来ました。
それは例の\ruby{猫大将}{ねこ|たい|しょう}でした。\\
「ワーッ。」とねずみはみんなちりぢり四方に逃げました。\\
「逃がさんぞ。コラッ。」と猫大将はその一匹を追いかけましたが、もうせまいすきまへずうっと深くもぐり込んでしまったので、
いくら猫大将が手をのばしてもとどきませんでした。

\newpage
\thispagestyle{fancy}
\fancyhead[R]{ここから P3}
\fancyhead[C]{\empty}
\indent 猫大将は「チェッ。」と舌打ちをして戻って来ましたが、クねずみのただ一匹しばられて残っているのを見て、びっくりして言いました。\\
「貴様はなんと言うものだ。」クねずみはもう落ち着いて答えました。\\
「クと申します。」\\
「フ、フ、そうか、なぜこんなにしているんだ。」\\
「暗殺されるためです。」\\
「フ、フ、フ。そうか。それはかあいそうだ。よしよし、おれが引き受けてやろう。おれのうちへ来い。
ちょうどおれのうちでは、子供が四人できて、それに家庭教師がなくて困っているところなんだ。来い。」\\
\indent 猫大将はのそのそ歩きだしました。\\
\indent クねずみはこわごわあとについて行きました。猫のおうちはどうもそれは立派なもんでした。
紫色の竹で編んであって中はわらや布きれでホクホクしていました。おまけにちゃあんとご飯を入れる道具さえあったのです。\\
\indent そしてその中に、\ruby{猫大将}{ねこ|たい|しょう}の子供が四人、やっと目をあいて、にゃあにゃあと鳴いておりました。\\
\indent 猫大将は子供らを一つずつなめてやってから言いました。\\
「お前たちはもう学問をしないといけない。ここへ先生をたのんで来たからな。よく習うんだよ。決して先生を食べてしまったりしてはいかんぞ。」

\newpage
\thispagestyle{fancy}
\fancyhead[R]{\empty}
\fancyhead[C]{P3}
\indent 子供らはよろこんでニヤニヤ笑って口々に、\\
「おとうさん、ありがとう。きっと習うよ。先生を食べてしまったりしないよ。」と言いました。\\
\indent クねずみはどうも思わず足がブルブルしました。\\
\indent 猫大将が言いました。\\
「教えてやってくれ。おもに算術をな。」\\
「へい。しょう、しょう、承知いたしました。」とクねずみが答えました。\\
\indent 猫大将はきげんよくニャーと鳴いてするりと向こうへ行ってしまいました。\\
\indent 子供らが叫びました。\\
「先生、早く算術を教えてください。先生。早く。」\\
\indent クねずみはさあ、これはいよいよ教えないといかんと思いましたので、口早に言いました。\\
「一に一をたすと二です。」\\
「そうだよ。」子供らが言いました。\\
「一から一を引くとなんにもなくなります。」\\
「わかったよ。」\\
\indent 子供らが叫びました。\\
「一に一をかけると一です。」\\
「きまってるよ。」と猫の子供らが目をりんと張ったまま答えました。\\
「一を一で割ると一です。」

\newpage
\thispagestyle{fancy}
\fancyhead[R]{\empty}
\fancyhead[C]{P3}
「それでいいよ。」と猫の子供らがよろこんで叫びました。そこでクねずみはすっかりのぼせてしまいました。\\
「一に二をたすと三です。」\\
「合ってるよ。」\\
「一から二を引くと……」と言おうとしてクねずみは、はっとつまってしまいました。\\
\indent すると猫の子供らは一度に叫びました。\\
「一から二は引かれないよ。」\\
\indent クねずみはあんまり猫の子供らがかしこいので、すっかりむしゃくしゃして、また早口に言いました。
そうでしょう。クねずみはいちばんはじめの一に一をたして二をおぼえるのに半年かかったのです。\\
「一に二をかけると二です。」\\
「そうともさ。」\\
「一を二で割ると……。」クねずみはまたつまってしまいました。すると猫の子供らはまた一度に声をそろえて、\\
「一割る二では半分だよ。」と叫びました。\\
\indent クねずみはあんまり\ruby{猫}{ねこ}の子供らの賢いのがしゃくにさわって、思わず「エヘン。エヘン。エイ。エイ。」\\
とやりました。すると猫の子供らは、しばらくびっくりしたように、顔を見合わせていましたが、
\newpage
\thispagestyle{fancy}
\fancyhead[R]{\empty}
\fancyhead[C]{P3 おわり}
やがてみんな一度に立ちあがって、\\
「なんだい。ねずめ、人をそねみやがったな。」と言いながらクねずみの足を一ぴきが一つずつかじりました。\\
\indent クねずみは非常にあわててばたばたして、急いで「エヘン、エヘン、エイ、エイ。」とやりましたがもういけませんでした。\\
\indent クねずみはだんだん四方の足から食われて行って、とうとうおしまいに四ひきの子猫は、
クねずみの胃の\ruby{腑}{ふ}のところで頭をコツンとぶっつけました。\\
\indent そこへ猫大将が帰って来て、\\
「何か習ったか。」とききました。\\
「ねずみをとることです。」と四ひきがいっしょに答えました。
\newpage
\nolinenumbers
底本:「童話集 銀河鉄道の夜 他十四編」谷川徹三編、岩波文庫、岩波書店
\\
\indent 1951(昭和26)年10月25日第1刷発行\\
\indent 1966(昭和41)年7月16日第18刷改版発行\\
\indent 2000(平成12)年5月25日第71刷発行\\
底本の親本:「宮沢賢治全集 第八巻」筑摩書房\\
\indent 1956(昭和31)年10月\\
入力:のぶ\\
校正:鈴木厚司\\
2003年8月3日作成\\
2008年2月29日修正\\
青空文庫作成ファイル:\\
このファイルは、インターネットの図書館、"http://www.aozora.gr.jp/">青空文庫(http://www.aozora.gr.jp/)で作られました。入力、校正、制作にあたったのは、ボランティアの皆さんです。\\
\\

\end{document}