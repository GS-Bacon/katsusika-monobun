\documentclass[
a4paper,
10pt,
book]
{tarticle}
\usepackage{daihon}


\begin{document}
\Large

%表紙
\thispagestyle{empty}
{\fontsize{30pt}{24pt}\selectfont
    蛙のゴム靴\\
    \begin{flushright}
        \ruby{宮沢}{みや|ざわ}\ruby{賢治}{けん|じ}\\
    \end{flushright}
}
\newpage
\pagewiselinenumbers
\thispagestyle{fancy}
\setcounter{page}{2}
\fancyhead[R]
{
    \begin{spacing}{0.6}
        \fontsize{11}{20pt}\selectfont
        {
            言語\\
            \indent\\
        }
    \end{spacing}

    \makebox[150mm][r]{
        \parbox<t>[b]{5zw}{\fontsize{10}{25.5pt}\selectfont{
                \vskip.15\baselineskip
                \rensuji{P1}\\\\
                \indent\\
                \indent\\
                \indent\\
                \indent\\
                \indent\\
                \indent\\
                \indent\\
                \indent\\
                \indent\\
                \indent\\
                \indent\\
                \indent\\
                \indent\\
                \indent\\
                \indent\\
                \indent\\
                \indent\\
            }}}\par
}
\nolinenumbers
\begin{linenumbers}
    \indent 松の木や\ruby{楢}{なら}の木の林の下を、深い\ruby{堰}{せき}が流れて\ruby{居}{を}りました。岸には\ruby{茨}{いばら}やつゆ草やたでが一杯にしげり、
    そのつゆくさの十本ばかり集った下のあたりに、カン\ruby{蛙}{がへる}のうちがありました。\\
    \indent それから、林の中の楢の木の下に、ブン蛙のうちがありました。\\
    \indent 林の向ふのすゝきのかげには、ベン蛙のうちがありました。\\
    \indent 三\ruby{疋}{びき}は年も同じなら大きさも大てい同じ、どれも負けず劣らず生意気で、いたづらものでした。\\
    \indent ある夏の暮れ方、カン蛙ブン蛙ベン蛙の三疋は、カン蛙の家の前のつめくさの広場に座って、雲見といふことをやって居りました。
    一体蛙どもは、みんな、夏の雲の峯を見ることが大すきです。
    じっさいあのまっしろなプクプクした、
    \ruby{玉髄}{ぎょく|ずゐ}のやうな、玉あられのやうな、又\ruby{蛋白石}{たん|ぱく|せき}を刻んでこさへた\ruby[g]{葡萄}{ぶだう}の置物のやうな雲の峯は、
    \ruby{誰}{たれ}の目にも立派に見えますが、蛙どもには殊にそれが見事なのです。\ruby{眺}{なが}めても眺めても\ruby{厭}{あ}きないのです。
    そのわけは、雲のみねといふものは、どこか蛙の頭の形に\ruby{肖}{に}てゐますし、それから春の蛙の卵に似てゐます。それで日本人ならば、
    丁度花見とか月見とかいふ\ruby{処}{ところ}を、蛙どもは雲見をやります。\\
    「どうも実に立派だね。だんだんペネタ形になるね。」\\
    「うん。うすい金色だね。永遠の生命を思はせるね。」\\
    「実に僕たちの理想だね。」\\
    \indent 雲のみねはだんだんペネタ形になって参りました。ペネタ形といふのは、蛙どもでは大へん\ruby{高尚}{かう|しゃう}なものになってゐます。
    平たいことなのです。雲の峰はだんだん崩れてあたりはよほどうすくらくなりました。\\
    「この\ruby{頃}{ごろ}、ヘロンの方ではゴム\ruby{靴}{ぐつ}がはやるね。」ヘロンといふのは蛙語です。人間といふことです。\\
    「うん。よくみんなはいてるやうだね。」
\end{linenumbers}

\newpage
\setcounter{page}{3}
\thispagestyle{fancy}
\fancyhead[R]
{


    \makebox[150mm][r]{
        \parbox<t>[b]{5zw}{\fontsize{10}{25.5pt}\selectfont{
                \vskip.15\baselineskip
                \indent \\ %1
                \indent \\ %2
                \indent \\ %3
                \indent \\ %4
                \indent \\ %5
                \indent \\ %6
                \indent \\ %7
                \indent \\ %8
                \indent \\ %9
                \indent \\ %10
                \rensuji{視覚}\rensuji{P2} \\ %11
                \indent \\ %12
                \indent \\ %13
                \indent \\ %14
                \indent \\ %15
                \indent \\ %16
                \indent \\ %17
                \indent \\ %18
            }}}\par
}
\begin{linenumbers}
    \noindent \,「僕たちもほしいもんだな。」\\
    「全くほしいよ。あいつをはいてなら\ruby{粟}{くり}のいがでも何でもこはくないぜ。」\\
    「ほしいもんだなあ。」\\
    「手に入れる工夫はないだらうか。」\\
    「ないわけでもないだらう。たゞ僕たちのはヘロンのとは大きさも型も大分ちがふから\ruby[<m>]{拵}{こしら}へ直さないと\ruby[g]{駄}{だ}\ruby{目}{め}だな。」\\
    「うん。それはさうさ。」\\
    \indent さて雲のみねは全くくづれ、あたりは\ruby{藍}{あゐ}色になりました。そこでベン蛙とブン蛙とは、\\
    「さよならね。」と\ruby{云}{い}ってカン蛙とわかれ、林の下の堰を勇ましく泳いで自分のうちに帰って行きました。
\end{linenumbers}

\indent \indent \indent \indent \indent \indent \indent \indent \indent \indent ※

\begin{linenumbers}
    \indent あとでカン\ruby{蛙}{がへる}は腕を組んで考へました。\ruby[g>]{桔梗}{ききゃう}色の\ruby{夕暗}{ゆふ|やみ}の中です。\\
    しばらくしばらくたってからやっと「ギッギッ」と二声ばかり鳴きました。そして草原をペタペ~タ歩いて畑にやって参りました、\\
    \indent それから声をうんと細くして、\\
    「\ruby[g]{野鼠}{のねずみ}さん、野鼠さん。まうし、まうし。」と呼びました。\\
    「ツン。」と野鼠は返事をして、ひょこりと蛙の前に出て来ました。そのうすぐろい顔も、もう見えないくらゐ暗いのです。
\end{linenumbers}

\newpage
\setcounter{page}{6}
\thispagestyle{fancy}
\fancyhead[R]{\empty}
\begin{linenumbers}
    \noindent \,「野鼠さん。今晩は。一つお前さんに頼みがあるんだが、きいて\ruby{呉}{く}れないかね。」\\
    「いや、それはきいてあげよう。去年の秋、僕が\ruby[<->]{蕎麦団子}{そ|ば|だん|ご}を食べて、チブスになって、ひどいわづらひをしたときに、あれほど親身の介抱を受けながら、その恩を何でわすれてしまふもんかね。」\\
    「さうか。そんなら一つお前さん、ゴム\ruby{靴}{ぐつ}を一足工夫して呉れないか。形はどうでもいいんだよ。僕がこしらへ直すから。」\\
    「あゝ、いゝとも。明日の晩までにはきっと持って来てあげよう。」\\
    「さうか。それはどうもありがたう。ではお願ひするよ。さよならね。」\\
    \indent カン蛙は大よろこびで自分のおうちへ帰って寝てしまひました。
\end{linenumbers}

\indent \indent \indent \indent \indent \indent \indent \indent \indent \indent ※

\begin{linenumbers}
    \noindent \, 次の晩方です。\\
    \indent カン蛙は又畑に来て、\\
    「野鼠さん。野鼠さん。まうし。まうし。」とやさしい声で呼びました。\\
    \indent 野鼠はいかにも疲れたらしく、目をとろんとして、はぁあとため息をついて、それに何だか大へん\ruby{憤}{おこ}って出て来ましたが、
    いきなり小さなゴム靴をカン蛙の前に投げ出しました。\\
    「そら、カン蛙さん。取ってお呉れ。ひどい難儀をしたよ。大へんな手数をしたよ。命がけで心配したよ。
    僕はお前のご恩はこれで払ったよ。少し払ひ過ぎた位かしらん。」と云ひながら、野鼠はぷいっと行ってしまったのでした。
\end{linenumbers}

\newpage
\setcounter{page}{7}
\thispagestyle{fancy}
\fancyhead[R]
{


    \makebox[150mm][r]{
        \parbox<t>[b]{5zw}{\fontsize{10}{25.5pt}\selectfont{
                \vskip.15\baselineskip
                \indent \\ %1
                \indent \\ %2
                \indent \\ %3
                \indent \\ %4
                \indent \\ %5
                \indent \\ %6
                \indent \\ %7
                \indent \\ %8
                \indent \\ %9
                \indent \\ %10
                \indent \\ %11
                \indent \\ %12
                \rensuji{聴覚}\rensuji{P3} \\ %13
                \indent \\ %14
                \indent \\ %15
                \indent \\ %16
                \indent \\ %17
                \indent \\ %18
            }}}\par
}
\begin{linenumbers}
    \indent カン蛙は、野鼠の\ruby{激昂}{げき|かう}のあんまりひどいのに、しばらくは\ruby{呆}{あき}れてゐましたが、
    なるほど考へて見ると、それも無理はありませんでした。まづ野鼠は、たゞの鼠にゴム靴をたのむ、
    たゞの鼠は\ruby{猫}{ねこ}にたのむ、猫は犬にたのむ、犬は馬にたのむ、馬は自分の\ruby{金沓}{かな|ぐつ}を\ruby{貰}{もら}ふとき、
    何とかかんとかごまかして、ゴム靴をもう一足受け取る、それから、馬がそれを犬に渡す、犬が猫に渡す、猫がたゞの鼠に渡す、
    たゞの鼠が野鼠に渡す、その渡しやうもいづれあとでお礼をよこせとか何とか、気味の悪い\ruby{語}{ことば}がついてゐたのでせう、
    そのほか馬はあとでゴム靴をごまかしたことがわかったら、人間からよほどひどい目にあはされるのでせう。
    それ全体を\ruby{野鼠}{の|ねずみ}が心配して考へるのですから、とても命にさはるほどつらい訳です。
    けれどもカン\ruby{蛙}{がへる}は、その立派なゴム\ruby{靴}{ぐつ}を見ては、もう\ruby{嬉}{うれ}しくて嬉しくて、口がむずむず云ふのでした。\\
    \indent 早速それを\ruby{叩}{たた}いたり引っぱったりして、丁度自分の足に合ふやうにこしらへ直し、にたにた笑ひながら足にはめ、その晩一ばん中歩きまはり、
    \ruby{暁方}{あけ|がた}になってから、ぐったり疲れて自分の家に帰りました。そして\ruby{睡}{ねむ}りました。
\end{linenumbers}

\indent \indent \indent \indent \indent \indent \indent \indent \indent \indent ※

\begin{linenumbers}
    「カン君、カン君、もう雲見の時間だよ。おいおい。カン君。」カン蛙は\ruby{眼}{め}をあけました。
    見るとブン蛙とベン蛙とがしきりに自分のからだをゆすぶってゐます。なるほど、
    東にはうすい\ruby[g]{黄金}{きん}色の雲の峯が美しく\ruby{聳}{そび}えてゐます。\\
    「や、君はもうゴム靴をはいてるね。どこから出したんだ。」\\
    「いや、これはひどい難儀をして大へんな手数をしてそれから命がけほど頭を痛くして取って来たんだ。
\end{linenumbers}

\newpage
\setcounter{page}{10}
\thispagestyle{fancy}
\fancyhead[R]
{


    \makebox[150mm][r]{
        \parbox<t>[b]{5zw}{\fontsize{10}{25.5pt}\selectfont{
                \vskip.15\baselineskip
                \indent \\ %1
                \indent \\ %2
                \indent \\ %3
                \indent \\ %4
                \indent \\ %5
                \indent \\ %6
                \indent \\ %7
                \indent \\ %8
                \indent \\ %9
                \indent \\ %10
                \indent \\ %11
                \indent \\ %12
                \indent \\ %13
                \indent \\ %14
                \indent \\ %15
                \indent \\ %16
                \indent \\ %17
                \indent \\ %18
            }}}\par
}
\begin{linenumbers}
    \noindent 君たちにはとても持てまいよ。歩いて見せようか。そら、いゝ\ruby[g]{工合}{ぐあひ}だらう。
    僕がこいつをはいてすっすっと歩いたらまるで芝居のやうだらう。まるでカーイのやうだらう、イーのやうだらう。」\\
    「うん、実にいゝね。僕たちもほしいよ。けれど仕方ないなあ。」\\
    「仕方ないよ。」\\
    \indent 雲の峯は銀色で、今が一番高い所です。けれどもベン蛙とブン蛙とは、雲なんかは見ないでゴム靴ばかり見てゐるのでした。\\
    \indent そのとき向ふの方から、一疋の美しいかへるの娘がはねて来てつゆくさの向ふからはづかしさうに顔を出しました。\\
    「ルラさん、今晩は。何のご用ですか。」\\
    「お父さんが、おむこさんを探して来いって。」娘の蛙は顔を少し平ったくしました。\\
    「僕なんかはどうかなあ。」ベン蛙が云ひました。\\
    「あるいは僕なんかもいゝかもしれないな。」ブン蛙が云ひました。\\
    \indent ところがカン蛙は一言も物を云はずに、すっすっとそこらを歩いてゐたばかりです。\\
    「あら、あたしもうきめたわ。」\\
    「\ruby{誰}{たれ}にさ?」二疋は眼をぱちぱちさせました。\\
    \indent カン蛙はまだすっすっと歩いてゐます。\\
    「あの方だわ。」娘の蛙は左手で顔をかくして右手の指をひろげてカン蛙を指しました。\\
    「おいカン君、お嬢さんがきみにきめたとさ。」
\end{linenumbers}

\newpage
\setcounter{page}{11}
\thispagestyle{fancy}
\fancyhead[R]
{


    \makebox[150mm][r]{
        \parbox<t>[b]{5zw}{\fontsize{10}{25.5pt}\selectfont{
                \vskip.15\baselineskip
                \indent \\ %1
                \indent \\ %2
                \indent \\ %3
                \indent \\ %4
                \indent \\ %5
                \indent \\ %6
                \indent \\ %7
                \indent \\ %8
                \indent \\ %9
                \indent \\ %10
                \indent \\ %11
                \indent \\ %12
                \indent \\ %13
                \indent \\ %14
                \indent \\ %15
                \indent \\ %16
                \indent \\ %17
                \indent \\ %18
            }}}\par
}
\begin{linenumbers}
    \noindent \,「何をさ?」\\
    \indent カン\ruby{蛙}{がへる}はけろんとした顔つきをしてこっちを向きました。\\
    「お嬢さんがおまへさんを連れて行くとさ。」\\
    \indent カン蛙は急いでこっちへ来ました。\\
    「お嬢さん今晩は、僕に何か用があるんですか。なるほど、さうですか。よろしい。承知しまし~た。\\
    \indent それで日はいつにしませう。式の日は。」\\
    「八月二日がいゝわ。」\\
    「それがいゝです。」カン蛙はすまして空を向きました。\\
    \indent そこでは雲の峯がいままたペネタ型になって流れてゐます。\\
    「そんならあたしうちへ帰ってみんなにさう云ふわ。」\\
    「えゝ、」\\
    「さよなら」\\
    「さよならね。」\\
    \indent ベン蛙とブン蛙はぶりぶり怒って、いきなりくるりとうしろを向いて帰ってしまひました。
    しゃくにさはったまぎれに、あの林の下の\ruby{堰}{せき}を、たゞ二足にちぇっちぇっと泳いだのでした。
    そのあとでカン蛙のよろこびやうと云ったらもうとてもありません。あちこちあるいてあるいて、
    東から二十日の月が登るころやっとうちに帰って寝ました。
\end{linenumbers}

\newpage
\setcounter{page}{14}
\thispagestyle{fancy}
\fancyhead[R]
{


    \makebox[150mm][r]{
        \parbox<t>[b]{5zw}{\fontsize{10}{25.5pt}\selectfont{
                \vskip.15\baselineskip
                \indent \\ %1
                \rensuji{言語}\rensuji{P4} \\ %2
                \indent \\ %3
                \indent \\ %4
                \indent \\ %5
                \indent \\ %6
                \indent \\ %7
                \indent \\ %8
                \indent \\ %9
                \indent \\ %10
                \indent \\ %11
                \indent \\ %12
                \indent \\ %13
                \indent \\ %14
                \indent \\ %15
                \indent \\ %16
                \indent \\ %17
                \indent \\ %18
            }}}\par
}
%\nolinenumbers
\indent \indent \indent \indent \indent \indent \indent \indent \indent \indent ※

\begin{linenumbers}
    \indent さてルラ蛙の方でも、いろいろ仕度をしたりカン蛙と談判をしたり、だんだん事がまとまりました。
    いよいよあさってが結婚式といふ日の明方、カン蛙は夢の中で、\\
    「今日は僕はどうしてもみんなの所を歩いて\ruby[g]{明後日}{あさって}の式に招待して来ないといけないな。」
    と云ひました。ところがその夜明方から朝にかけて、いよいよ雨が降りはじめました。林はガアガアと鳴り、
    カン蛙のうちの前のつめくさは、うす濁った水をかぶってぼんやりとかすんで見えました。それでもカン蛙は勇んで家を出ました。
    せきの水は濁って大へんに増し、幾本もの\ruby{蓼}{たで}やつゆくさは、すっかり水の中になりました、
    飛び込むのは\ruby[g]{一寸}{ちょっと}こはいくらゐです。カン蛙は、けれども一本のたでから、ピチャンと水に飛び込んで、
    ツイツイツイツイ泳ぎました。泳ぎながらどんどん流されました。それでもとにかく向ふの岸にのぼりました。\\
    \indent それから\ruby{苔}{こけ}の上をずんずん通り、幾本もの虫のあるく道を横切って、
    大粒の雨にうたれゴム\ruby{靴}{ぐつ}をピチャピチャ云はせながら、\ruby{楢}{なら}の木の下のブン蛙のおうちに来て高く叫びました。\\
    「今日は、今日は。」\\
    「どなたですか。あゝ君か。はひり\ruby{給}{たま}へ。」\\
    「うん、どうもひどい雨だね。パッセン大街道も今日はいきものの影さへないぞ。」\\
    「さうか。ずゐぶんひどい雨だ。」\\
    「ところで君も知ってる通り、\ruby[g]{明後日}{あさって}は僕の結婚式なんだ。どうか来て呉れ給へ。」
\end{linenumbers}

\newpage
\setcounter{page}{15}
\thispagestyle{fancy}
\fancyhead[R]
{


    \makebox[150mm][r]{
        \parbox<t>[b]{5zw}{\fontsize{10}{25.5pt}\selectfont{
                \vskip.15\baselineskip
                \indent \\ %1
                \indent \\ %2
                \indent \\ %3
                \indent \\ %4
                \indent \\ %5
                \indent \\ %6
                \indent \\ %7
                \indent \\ %8
                \indent \\ %9
                \indent \\ %10
                \indent \\ %11
                \indent \\ %12
                \indent \\ %13
                \indent \\ %14
                \indent \\ %15
                \indent \\ %16
                \indent \\ %17
                \indent \\ %18
            }}}\par
}

\begin{linenumbers}
    「うん。さうさう。さう云へばあの時あのちっぽけな赤い虫が何かそんなこと云ってゐたやうだったね。行かう。」\\
    「ありがたう。どうか頼むよ。それではさよならね。」\\
    「さよならね。」\\
    \indent カン\ruby{蛙}{がへる}は又ピチャピチャ林の中を通ってすゝきの中のベン蛙のうちにやって参りました。\\
    「今日は、今日は。」\\
    「どなたですか。あゝ君か。はひれ。」\\
    「ありがたう。どうもひどい雨だ。パッセン大街道も今日はしんとしてるよ。」\\
    「さうか。ずゐぶんひどいね。」\\
    「ところで君も知ってるだらうが明後日僕の結婚式なんだ。どうか来て呉れ給へ。」\\
    「あゝ、そんなことどこかで聞いたっけねい。行かう。」\\
    「どうか。ではさよならね。」\\
    「さよならね。」そしてカン蛙は又ピチャピチャ林の中を歩き、プイプイ\ruby{堰}{せき}を泳いで、おうちに帰ってや\\っと安心しました。
\end{linenumbers}

\indent \indent \indent \indent \indent \indent \indent \indent \indent \indent ※

\begin{linenumbers}
    次の日にブン蛙はベン蛙のところへやって来たのでした。\\
    「今日は、今日は。」
\end{linenumbers}

\newpage
\setcounter{page}{18}
\thispagestyle{fancy}
\fancyhead[R]
{


    \makebox[150mm][r]{
        \parbox<t>[b]{5zw}{\fontsize{10}{25.5pt}\selectfont{
                \vskip.15\baselineskip
                \indent \\ %1
                \indent \\ %2
                \indent \\ %3
                \indent \\ %4
                \indent \\ %5
                \indent \\ %6
                \indent \\ %7
                \indent \\ %8
                \indent \\ %9
                \indent \\ %10
                \indent \\ %11
                \indent \\ %12
                \indent \\ %13
                \indent \\ %14
                \indent \\ %15
                \indent \\ %16
                \rensuji{視覚}\rensuji{P5} \\ %17
                \indent \\ %18
            }}}\par
}
\begin{linenumbers}
    \noindent \,「はい。やあ、君か。はひれ。」\\
    「カンが来たらう。」\\
    「うん。いまいましいね。」\\
    「全くだ。畜生。何とかひどい目にあはしてやりたいね。」\\
    「僕がうまいこと考へたよ。明日の朝ね、雨がはれたら結婚式の前に\ruby[g]{一寸}{ちょっと}散歩しようと云って\\
    \indent あいつを引っぱり出して、あそこの\ruby{萱}{かや}の刈跡をあるくんだよ。僕らも少しは痛いだらうがま~あ\\
    \indent 我慢してさ。
    するとあいつのゴム\ruby{靴}{ぐつ}がめちゃめちゃになるだらう。」\\
    「うん。それはいゝね。しかし僕はまだそれ位ぢゃ腹が\ruby{癒}{い}えないよ。
    結婚式がすんだらあいつら\\
    \indent を引っぱり出して、あの畑の麦をほした\ruby{杭}{くひ}の穴に落してやりたいね。上に何か木の葉でも\\
    \indent かぶせて置かう。それは僕がやって置くよ。面白いよ。」\\
    「それもいゝね。ぢゃ、雨がはれたらね。」\\
    「うん。」\\
    「ではさよならね。」\\
    \indent \ruby{蛙}{かへる}の\ruby[g]{挨拶}{あいさつ}の「さよならね」ももう鼻について\ruby{厭}{あ}きて参りました。
    もう少しです。我慢して下さい。ほんのもう少しですから。
\end{linenumbers}

\indent \indent \indent \indent \indent \indent \indent \indent \indent \indent ※

\begin{linenumbers}
    \indent 次の日のひるすぎ、雨がはれて\ruby{陽}{ひ}が\ruby{射}{さ}しました。ベン蛙とブン蛙とが一緒にカン蛙のうちへやって来ま
\end{linenumbers}

\newpage
\setcounter{page}{19}
\thispagestyle{fancy}
\fancyhead[R]
{
    \makebox[150mm][r]{
        \parbox<t>[b]{5zw}{\fontsize{10}{25.5pt}\selectfont{
                \vskip.15\baselineskip
                \indent \\ %1
                \indent \\ %2
                \indent \\ %3
                \indent \\ %4
                \indent \\ %5
                \indent \\ %6
                \indent \\ %7
                \indent \\ %8
                \indent \\ %9
                \indent \\ %10
                \indent \\ %11
                \indent \\ %12
                \indent \\ %13
                \indent \\ %14
                \indent \\ %15
                \indent \\ %16
                \indent \\ %17
                \indent \\ %18
            }}}\par
}
\begin{linenumbers}
    した。\\
    「やあ、今日はおめでたう。お招き通りやって来たよ。」\\
    「うん、ありがたう。」\\
    「ところで式まで大分時間があるだらう。少し歩かうか。散歩すると血色がよくなるぜ。」\\
    「さうだ。では行かう。」\\
    「三人で手をつないでかうね。」ブン蛙とベン蛙とが両方からカン蛙の手を取りました。\\
    「どうも雨あがりの空気は、実にうまいね。」\\
    「うん。さっぱりして気持ちがいゝね。」三疋は\ruby{萱}{かや}の刈跡にやって参りました。\\
    「あゝいゝ景色だ。こゝを通って行かう。」\\
    「おい。こゝはよさうよ。もう帰らうよ。」\\
    「いゝや折角来たんだもの。も少し行かう。そら歩きたまへ。」二疋は両方からぐいぐいカン蛙\\
    \indent の手をひっぱって、
    自分たちも足の痛いのを我慢しながらぐんぐん萱の刈跡をあるきました。\\
    「おい。よさうよ。よして呉れよ。こゝは歩けないよ。あぶないよ。帰らうよ。」\\
    「実にいゝ景色だねえ。も少し急いで行かうか。」と二疋が両方から、まだ破けないカン蛙のゴム\ruby{靴}{ぐつ}を見ながら一緒に云ひました。\\
    「おい。よさうよ。冗談じゃない。よさう。あ痛っ。あぁあ、たうとう穴があいちゃった。」
    「どうだ。この空気のうまいこと。」\\
    「おい。帰らうよ。ひっぱらないで呉れよ。」
\end{linenumbers}

\newpage
\setcounter{page}{22}
\thispagestyle{fancy}
\fancyhead[R]
{


    \makebox[150mm][r]{
        \parbox<t>[b]{5zw}{\fontsize{10}{25.5pt}\selectfont{
                \vskip.15\baselineskip
                \indent \\ %1
                \indent \\ %2
                \indent \\ %3
                \indent \\ %4
                \indent \\ %5
                \indent \\ %6
                \indent \\ %7
                \indent \\ %8
                \indent \\ %9
                \indent \\ %10
                \indent \\ %11
                \indent \\ %12
                \indent \\ %13
                \indent \\ %14
                \indent \\ %15
                \indent \\ %16
                \indent \\ %17
                \indent \\ %18
            }}}\par
}
\begin{linenumbers}
    \noindent \,「実にいゝ景色だねえ。」\\
    「放して呉れ。放して呉れ。放せったら。畜生。」\\
    「おや、君は何かに足をかじられたんだね。そんなにもがかなくてもいゝよ。しっかり押へてる\\
    \indent から。」\\
    「放せ、放せ、放せったら、畜生。」\\
    「まだかじってるかい。そいつは大変だ。早く逃げ給へ。走らう。さあ。そら。」\\
    「痛いよ。放せったら放せ。えい畜生。」\\
    「早く、早く。そら、もう大丈夫だ。おや。君の\ruby{靴}{くつ}がぼろぼろだね。どうしたんだらう。」\\
    \indent 実際ゴム靴はもうボロボロになって、カン\ruby{蛙}{がへる}の足からあちこちにちらばって、無くなりまし~た。\\
    \indent カン蛙は何とも言へないうらめしさうな顔をして、口をむにゃむにゃやりました。実はこれは歯を食ひしばるところなのですが、
    歯がないのですからむにゃむにゃやるより仕方ないのです。二疋はやっと手をはなして、しきりに両方からお世辞を云ひました。\\
    「君、あんまり力を落さない方がいゝよ。靴なんかもうあったってないったって、お嫁さんは来るんだから。」\\
    「もう時間だらう。帰らう。帰って待ってようか。ね。君。」\\
    \indent カン蛙はふさぎこみながらしぶしぶあるき出しました。
\end{linenumbers}

\indent \indent \indent \indent \indent \indent \indent \indent \indent \indent ※

\begin{linenumbers}
    \indent 三疋がカン蛙のおうちに着いてから、しばらくたって、ずうっと向ふから、\ruby{蕗}{ふき}の葉をかざしたりがまの
\end{linenumbers}

\newpage
\setcounter{page}{23}
\thispagestyle{fancy}
\fancyhead[R]
{


    \makebox[150mm][r]{
        \parbox<t>[b]{5zw}{\fontsize{10}{25.5pt}\selectfont{
                \vskip.15\baselineskip
                \indent \\ %1
                \indent \\ %2
                \indent \\ %3
                \indent \\ %4
                \indent \\ %5
                \indent \\ %6
                \indent \\ %7
                \indent \\ %8
                \indent \\ %9
                \indent \\ %10
                \indent \\ %11
                \indent \\ %12
                \indent \\ %13
                \indent \\ %14
                \indent \\ %15
                \indent \\ %16
                \rensuji{聴覚}\rensuji{P6} \\ %17
                \indent \\ %18
            }}}\par
}
\begin{linenumbers}
    穂を立てたりしてお嫁さんの行列がやって参りました。\\
    \indent だんだん近くになりますと、お父さんにあたるがん郎がへるが、\\
    「こりゃ、むすめ、むこどのはあの三人の中のどれぢゃ。」とルラ蛙をふりかへってたづねまし~た。\\
    \indent ルラ蛙は、小さな目をパチパチさせました。といふわけは、はじめカン蛙を見たときは、実はゴム靴のほかにはなんにも気を付けませんでしたので、
    三疋ともはだしでぞろりとならんでゐるのでは実際どうも困ってしまひました。そこで仕方なく、\\
    「もっと向ふへ行かないと、よくわからないわ。」と云ひました。\\
    「さうですとも。間違っては大へんです。よくおちついて。」と\ruby[g]{仲人}{なかうど}のかへるもうしろで云ひ~ま~し~た。\\
    \indent ところがもっと近くによりますと、\ruby{尚更}{なほ|さら}わからなくなりました。三疋とも口が大きくて、うすぐろくて、
    眼の出た\ruby[g]{工合}{ぐあひ}も実によく似てゐるのです。これにはいよいよどうも困ってしまったのでした。
    ところが、そのうちに、一番右はじに居たカン蛙がパクッと口をあけて、一足前に出ておじぎをしました。そこでルラ蛙もやっと安心して、\\
    「あの方よ。」と云ひました。さてそれから式がはじまりました。その式の盛大なこと酒もりの立派なこととても書くのも大へんです。\\
    \indent とにかく式がすんで、向ふの方はみな引きあげて行きました。その時丁度雲のみねが一番か~ゞやいて居りました。\\
    「さあ新婚旅行だ。」とベン蛙が云ひました。\\
    「僕たちはぢきそこまで見送らう。」ブン蛙が云ひました。
\end{linenumbers}

\newpage
\setcounter{page}{26}
\thispagestyle{fancy}
\fancyhead[R]
{


    \makebox[150mm][r]{
        \parbox<t>[b]{5zw}{\fontsize{10}{25.5pt}\selectfont{
                \vskip.15\baselineskip
                \indent \\ %1
                \indent \\ %2
                \indent \\ %3
                \indent \\ %4
                \indent \\ %5
                \indent \\ %6
                \indent \\ %7
                \indent \\ %8
                \indent \\ %9
                \indent \\ %10
                \indent \\ %11
                \indent \\ %12
                \indent \\ %13
                \indent \\ %14
                \indent \\ %15
                \indent \\ %16
                \indent \\ %17
                \indent \\ %18
            }}}\par
}
\begin{linenumbers}
    \indent カン\ruby{蛙}{がへる}も仕方なく、ルラ蛙もつれて、新婚旅行に出かけました。そしてたちまちあの木の葉をかぶせた\ruby{杭}{くひ}あとに来たのです。
    ブン蛙とベン蛙が、\\
    「あゝ、こゝはみちが悪い。おむこさん。手を引いてあげよう。」と云ひながら、カン蛙が急いでちゞめる間もなく、両方から手をとって、
    自分たちは穴の両側を歩きながら無理にカン蛙を穴の上にひっぱり出しました。するとカン蛙の載った木の葉がガサリと鳴り、
    カン蛙はふらふらっと一寸ばかりめり込みました。ブン蛙とベン蛙がくるりと外の方を向いて逃げようとしましたが、
    カン蛙がピタリと両方共とりついてしまひましたので二疋のふんばった足がぷるぷるっとけいれんし、そのつぎにはたうとう「ポトン、バチャン。」\\
    \indent 三疋とも、杭穴の底の泥水の中に\ruby{陥}{お}ちてしまひました。上を見ると、まるで小さな円い空が見えるだけ、
    かゞやく雲の峯は\ruby[g]{一寸}{ちょっと}のぞいて居りますが、蛙たちはもういくらもがいてもとりつくものもありませんでした。\\
    \indent そこでルラ蛙はもう昔習った六百\ruby{米}{メートル}の奥の手を出して一目散にお父さんのところへ走って行きました。
    するとお父さんたちはお酒に酔ってゐてみんなぐうぐう\ruby{睡}{ねむ}ってゐていくら起しても起きませんでした。
    そこでルラ蛙はまたもとのところへ走って来てまはりをぐるぐるぐるぐるまはって泣きました。\\
    \indent そのうちだんだん夜になりました。\\
    \indent \indent \indent \indent パチャパチャパチャパチャ。\\
    \indent ルラ蛙はまたお父さんのところへ行きました。\\
    \indent いくら起しても起きませんでした。
\end{linenumbers}

\newpage
\setcounter{page}{27}
\thispagestyle{fancy}
\fancyhead[R]
{
    \makebox[150mm][r]{
        \parbox<t>[b]{5zw}{\fontsize{10}{25.5pt}\selectfont{
                \vskip.15\baselineskip
                \indent \\ %1
                \indent \\ %2
                \indent \\ %3
                \indent \\ %4
                \indent \\ %5
                \indent \\ %6
                \indent \\ %7
                \indent \\ %8
                \indent \\ %9
                \indent \\ %10
                \indent \\ %11
                \indent \\ %12
                \indent \\ %13
                \indent \\ %14
                \indent \\ %15
                \indent \\ %16
                \indent \\ %17
                \indent \\ %18
            }}}\par
}
\begin{linenumbers}
    \indent 夜があけました。\\
    \indent \indent \indent \indent パチャパチャパチャパチャ。\\
    \indent ルラ蛙はまたお父さんのところへ行きました。\\
    \indent いくら起しても起きませんでした。\\
    \indent 日が暮れました。雲のみねの頭。\\
    \indent \indent \indent \indent パチャパチャパチャパチャ。
    \indent ルラ蛙はまたお父さんのところへ行きました。\\
    \indent いくら起しても起きませんでした。\\
    \indent 夜が明けました。\\
    \indent \indent \indent \indent パチャパチャパチヤパチャ。\\
    \indent 雲のみね。ペネタ形。\\
    \indent ちゃうどこのときお父さんの蛙はやっと眼がさめてルラ蛙がどうなったか見ようと思って出掛けて来ました。\\
    \indent するとそこにはルラ\ruby{蛙}{がへる}がつかれてまっ青になって腕を胸に組んで座ったまゝ\ruby{睡}{ねむ}ってゐまし~た。\\
    「おいどうしたのか。おい。」\\
    「あらお父さん、三人この中へおっこってゐるわ。もう死んだかもしれないわ」\\
    \indent お父さんの蛙は落ちないやうに気をつけながら耳を穴の口へつけて音をききましたら、かすかにぴちゃといふ音がしました。
\end{linenumbers}

\newpage
\setcounter{page}{30}
\thispagestyle{fancy}
\fancyhead[R]
{


    \makebox[150mm][r]{
        \parbox<t>[b]{5zw}{\fontsize{10}{25.5pt}\selectfont{
                \vskip.15\baselineskip
                \indent \\ %1
                \indent \\ %2
                \indent \\ %3
                \indent \\ %4
                \indent \\ %5
                \indent \\ %6
                \indent \\ %7
                \indent \\ %8
                \indent \\ %9
                \indent \\ %10
                \indent \\ %11
                \indent \\ %12
                \indent \\ %13
                \indent \\ %14
                \indent \\ %15
                \indent \\ %16
                \indent \\ %17
                \indent \\ %18
            }}}\par
}
\begin{linenumbers}
    \noindent「占めた」と叫んでお父さんは急いで帰って仲間の蛙をみんなつれて来ました。そして林の中からひかげのかつらをとって来てそれを穴の中につるして、
    たうとう一ぴきづつ穴からひきあげました。\\
    \indent 三疋とももう白い腹を上へ向けて眼はつぶって口も堅くしめて半分死んでゐました。\\
    \indent みんなでごまざいの毛をとって来てこすってやったりいろいろしてやっと助けました。\\
    \indent そこでカン蛙ははじめてルラ蛙といっしょになりほかの蛙も大へんそれからは心を改めてみんなよく働くやうになりました。
\end{linenumbers}
    \newpage
底本:「新修宮沢賢治全集\indent 第十一巻」筑摩書房\\
\indent \indent \indent 1979(昭和54)年11月15日初版第1刷発行\\
\indent \indent \indent 1983(昭和58)年12月20日初版第5刷発行\\
※底本は旧仮名ですが、拗促音は小書きされています。これにならい、ルビの拗促音も、小書きにしました。\\
入力:林\indent 幸雄\\
校正:土屋隆\\
2008年2月27日作成\\
2008年11月30日修正\\
青空文庫作成ファイル:\\
このファイルは、インターネットの図書館、青空文庫(http://www.aozora.gr.jp/)で作られました。入力、校正、制作にあたったのは、ボランティアの皆さんです。\\
●表記について\\
•	このファイルは W3C 勧告 XHTML1.1 にそった形式で作成されています。\\
\begin{comment}
a
\end{comment}
\end{document}