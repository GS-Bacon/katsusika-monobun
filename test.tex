\documentclass[a4paper,
tate,
book]
{jlreq}
\usepackage{pxrubrica}
\usepackage{otf}
\pagestyle{myheadings}
\begin{document}
 \ruby[g]{北守}{ほくしゅ} \ruby[g]{将軍}{しょうぐん}と \ruby[g]{三人}{さんにん} \ruby[g]{兄弟}{きょうだい}の \ruby[g]{医者}{いしゃ}

\ruby[g]{宮沢}{みやざわ} \ruby[g]{賢治}{けんじ}

             \ruby[g]{一}{いち}、 \ruby[g]{三人}{さんにん} \ruby[g]{兄弟}{きょうだい}の \ruby[g]{医者}{いしゃ}

  むかしラユーといふ \ruby[g]{首都}{しゅと}に、 \ruby[g]{兄弟}{きょうだい} \ruby[g]{三人}{さんにん}の \ruby[g]{医者}{いしゃ}が \ruby[g]{ゐ}{い}た。いちばん \ruby[g]{上}{うえ}のリンパーは、 \ruby[g]{普通}{ふつう}の \ruby[g]{人}{ひと}の \ruby[g]{医者}{いしゃ}だつた。その \ruby[g]{弟}{おとうと}のリンプーは、 \ruby[g]{馬}{うま}や \ruby[g]{羊}{ひつじ}の \ruby[g]{医者}{いしゃ}だつた。いちばん \ruby[g]{末}{すえ}のリンポーは、 \ruby[g]{草}{くさ}だの \ruby[g]{木}{き}だのの \ruby[g]{医者}{いしゃ}だつた。そして \ruby[g]{兄弟}{きょうだい} \ruby[g]{三人}{さんにん}は、 \ruby[g]{町}{まち}のいちばん \ruby[g]{南}{みなみ}にあたる、 \ruby[g]{黄}{き}いろな \ruby[g]{崖}{がけ}のとつぱなへ、 \ruby[g]{青い}{あおい} \ruby[g]{瓦}{かわら}の \ruby[g]{病院}{びょういん}を、 \ruby[g]{三}{みっ}つならべて \ruby[g]{建}{た}てて \ruby[g]{ゐ}{い}て、てんでに \ruby[g]{白}{しろ}や \ruby[g]{朱}{しゅ}の \ruby[g]{旗}{はた}を、 \ruby[g]{風}{ふう}にぱたぱた \ruby[g]{云}{い}はせて \ruby[g]{ゐ}{い}た。
   \ruby[g]{坂}{さか}のふもとで \ruby[g]{見}{み}て \ruby[g]{ゐ}{い}ると、 \ruby[g]{漆}{うるし}にかぶれた \ruby[g]{坊}{ぼう}さんや、 \ruby[g]{少}{すこ}しびつこをひく \ruby[g]{馬}{うま}や、 \ruby[g]{萎}{しお}れかかつた \ruby[g]{牡丹}{ぼたん}の \ruby[g]{鉢}{はち}を、 \ruby[g]{車}{くるま}につけて \ruby[g]{引}{ひ}く \ruby[g]{園丁}{えんてい}や、いんこを \ruby[g]{入}{はい}れた \ruby[g]{鳥籠}{とりかご}や、 \ruby[g]{次}{つぎ}から \ruby[g]{次}{つぎ}とのぼつて \ruby[g]{行}{い}つて、さて \ruby[g]{坂上}{さかがみ}に \ruby[g]{行}{い}き \ruby[g]{着}{き}くと、 \ruby[g]{病気}{びょうき}の \ruby[g]{人}{ひと}は、 \ruby[g]{左}{ひだり}のリンパー \ruby[g]{先生}{せんせい}へ、 \ruby[g]{馬}{うま}や \ruby[g]{羊}{ひつじ}や \ruby[g]{鳥類}{ちょうるい}は、 \ruby[g]{中}{なか}のリンプー \ruby[g]{先生}{せんせい}へ、 \ruby[g]{草木}{くさき}をもつた \ruby[g]{人}{ひと}たちは、 \ruby[g]{右}{みぎ}のリンポー \ruby[g]{先生}{せんせい}へ、 \ruby[g]{三}{みっ}つにわかれてはひるのだつた。
  さて \ruby[g]{三人}{さんにん}は \ruby[g]{三人}{さんにん}とも、 \ruby[g]{実}{じつ}に \ruby[g]{医術}{いじゅつ}もよくできて、また \ruby[g]{仁心}{じんしん}も \ruby[g]{相当}{そうとう}あつて、たしかにもはや \ruby[g]{名医}{めいい}の \ruby[g]{類}{たぐい}であつたのだが、まだいゝ \ruby[g]{機会}{おり}がなかつたために \ruby[g]{別}{べつ}に \ruby[g]{位}{くらい}もなかつたし、 \ruby[g]{遠}{とお}くへ \ruby[g]{名前}{なまえ}も \ruby[g]{聞}{きこ}なかつた。ところがたうとうある \ruby[g]{日}{ひ}のこと、ふしぎなことが \ruby[g]{起}{おこ}つてきた。

             \ruby[g]{二}{に}、 \ruby[g]{北守}{ほくしゅ} \ruby[g]{将軍}{しょうぐん}ソンバーユー

  ある \ruby[g]{日}{ひ}のちやうど \ruby[g]{日}{ひ}の \ruby[g]{出}{だ}ごろ、ラユーの \ruby[g]{町}{まち}の \ruby[g]{人}{ひと}たちは、はるかな \ruby[g]{北}{きた}の \ruby[g]{野原}{のはら}の \ruby[g]{方}{ほう}で、 \ruby[g]{鳥}{とり}か \ruby[g]{何}{なに}かがたくさん \ruby[g]{群}{む}れて、 \ruby[g]{声}{こえ}をそろへて \ruby[g]{鳴}{な}くやうな、をかしな \ruby[g]{音}{おと}を、ときどき \ruby[g]{聴}{き}いた。はじめは \ruby[g]{誰}{だれ}も \ruby[g]{気}{き}にかけず、 \ruby[g]{店}{みせ}を \ruby[g]{掃}{は}いたりして \ruby[g]{ゐ}{い}たが、 \ruby[g]{朝}{あさ}めしすこしすぎたころ、だんだんそれが \ruby[g]{近}{ちか}づいて、みんな \ruby[g]{立派}{りっぱ}なチヤルメラや、ラツパの \ruby[g]{音}{おと}だとわかつてくると、 \ruby[g]{町}{まち}ぢゆうにはかにざわざわした。その \ruby[g]{間}{あいだ}にはぱたぱたいふ、 \ruby[g]{太鼓}{たいこ}の \ruby[g]{類}{たぐい}の \ruby[g]{音}{おと}もする。もう \ruby[g]{商人}{しょうにん}も \ruby[g]{職人}{しょくにん}も、 \ruby[g]{仕事}{しごと}がすこしも \ruby[g]{手}{て}につかない。 \ruby[g]{門}{もん}を \ruby[g]{守}{まも}つた \ruby[g]{兵隊}{へいたい}たちは、まづ \ruby[g]{門}{もん}をみなしつかりとざし、 \ruby[g]{町}{まち}をめぐつた \ruby[g]{壁}{かべ}の \ruby[g]{上}{うえ}には、 \ruby[g]{見張}{みは}りの \ruby[g]{者}{もの}をならべて \ruby[g]{置}{お}いて、それからお \ruby[g]{宮}{みや}へ \ruby[g]{知}{しら}らせを \ruby[g]{出}{だ}した。
  そしてその \ruby[g]{日}{ひ}の \ruby[g]{午}{ひる}ちかく、ひづめの \ruby[g]{音}{おと}や \ruby[g]{鎧}{よろひ}の \ruby[g]{気配}{けはい}、また \ruby[g]{号令}{ごうれい}の \ruby[g]{声}{こえ}もして、 \ruby[g]{向}{むこ}ふはすつかり、この \ruby[g]{町}{まち}を、 \ruby[g]{囲}{かこ}んでしまつた \ruby[g]{模様}{もよう}であつた。
   \ruby[g]{番兵}{ばんへい}たちや、あらゆる \ruby[g]{町}{まち}の \ruby[g]{人}{ひと}たちが、まるでどきどきやりながら、 \ruby[g]{矢}{や}を \ruby[g]{射}{い}る \ruby[g]{孔}{あな}からのぞいて \ruby[g]{見}{み}た。 \ruby[g]{壁}{かべ}の \ruby[g]{外}{そと}から \ruby[g]{北}{きた}の \ruby[g]{方}{ほう}、まるで \ruby[g]{雲霞}{うんか}の \ruby[g]{軍勢}{ぐんぜい}だ。ひらひらひかる \ruby[g]{三角旗}{さんかくばた}や、ほこがさながら \ruby[g]{林}{はやし}のやうだ。ことになんとも \ruby[g]{奇体}{きたい}なことは、 \ruby[g]{兵隊}{へいたい}たちが、みな \ruby[g]{灰}{はい}いろでぼさぼさして、なんだかけむりのやうなのだ。するどい \ruby[g]{眼}{め}をして、ひげが \ruby[g]{二}{ふた}いろまつ \ruby[g]{白}{しろ}な、せなかのまがつた \ruby[g]{大将}{たいしょう}が、 \ruby[g]{尻尾}{しつぽ}が \ruby[g]{箒}{はうき}のかたちになつて、うしろにぴんとのびて \ruby[g]{ゐ}{い}る \ruby[g]{白馬}{はくば}に \ruby[g]{乗}{の}つて \ruby[g]{先頭}{せんとう}に \ruby[g]{立}{た}ち、 \ruby[g]{大}{おお}きな \ruby[g]{剣}{けん}を \ruby[g]{空}{そら}にあげ、 \ruby[g]{声}{こえ} \ruby[g]{高々}{たかだか}と \ruby[g]{歌}{うた}つて \ruby[g]{ゐ}{い}る。

「 \ruby[g]{北守}{ほくしゅ} \ruby[g]{将軍}{しょうぐん}ソンバーユーは
いま \ruby[g]{塞外}{さいがい}の \ruby[g]{砂漠}{さばく}から
やつとのことで \ruby[g]{戻}{もど}つてきた。
 \ruby[g]{勇}{いさ}ましい \ruby[g]{凱旋}{がいせん}だと \ruby[g]{云}{い}ひたいが
 \ruby[g]{実}{じつ}はすつかり \ruby[g]{参}{まい}つて \ruby[g]{来}{き}たのだ
とにかくあすこは \ruby[g]{寒}{さむ}い \ruby[g]{処}{ところ}さ。
 \ruby[g]{三十年}{さんじゅうねん}といふ \ruby[g]{黄}{き}いろなむかし
おれは \ruby[g]{十万}{じゅうまん}の \ruby[g]{軍勢}{ぐんぜい}をひき \ruby[g]{ゐ}{い}
この \ruby[g]{門}{もん}をくぐつて \ruby[g]{威}{い} \ruby[g]{張}{ば}つて \ruby[g]{行}{い}つた。
それからどうだもう \ruby[g]{見}{み}るものは \ruby[g]{空}{そら}ばかり
 \ruby[g]{風}{かぜ}は \ruby[g]{乾}{かわ}いて \ruby[g]{砂}{すな}を \ruby[g]{吹}{ふ}き
 \ruby[g]{雁}{かり}さへ \ruby[g]{干}{ほ}せてたびたび \ruby[g]{落}{おと}ちた
おれはその \ruby[g]{間}{あいだ}  \ruby[g]{馬}{うま}でかけ \ruby[g]{通}{とお}し
 \ruby[g]{馬}{うま}がつかれてたびたびペタンと \ruby[g]{座}{すわ}り
 \ruby[g]{涙}{なみだ}をためてはじつと \ruby[g]{遠}{とお}くの \ruby[g]{砂}{すな}を \ruby[g]{見}{み}た。
その \ruby{度}{たび}ごとにおれは \ruby[g]{鎧}{よろひ}のかくしから
 \ruby{塩}{しお}をすこうし \ruby[g]{取}{と}り \ruby[g]{出}{だ}して
 \ruby{馬}{うま}に \ruby[g]{嘗}{な}めさせては \ruby[g]{元気}{げんき}をつけた。
その \ruby{馬}{うま}も \ruby[g]{今}{いま}では \ruby[g]{三十五歳}{さんじゅうごさい}
 \ruby[g]{五里}{ごり}かけるにも \ruby[g]{四時間}{よじかん}かゝる
それからおれはもう \ruby[g]{七十}{しちじゅう}だ。
とても \ruby{帰}{かえ}れまいと \ruby[g]{思}{おも}つて \ruby[g]{ゐ}{い}たが
ありがたや \ruby[g]{敵}{てき}が \ruby[g]{残}{のこ}らず \ruby[g]{脚気}{かっけ}で \ruby[g]{死}{し}んだ
 \ruby[g]{今年}{ことし}の \ruby[g]{夏}{なつ}はへんに \ruby[g]{湿気}{しっけ}が \ruby[g]{多}{おお}かつたでな。
それに \ruby[g]{脚気}{かっけ}の \ruby[g]{原因}{げんいん}が
あんまりこつちを \ruby[g]{追}{お}ひかけて
 \ruby[g]{砂}{すな}を \ruby[g]{走}{はし}つたためなんだ
さうしてみればどうだやつぱり \ruby[g]{凱旋}{がいせん}だらう。
 \ruby[g]{殊}{こと}にも \ruby[g]{一}{ひと}つほめられて \ruby[g]{いゝ}{いい}ことは
 \ruby[g]{十万人}{じゅうまんにん}もでかけたものが
 \ruby[g]{九万人}{きゅうまんにん}まで \ruby[g]{戻}{もど}つて \ruby[g]{来た}{き  }。
 \ruby[g]{死}{しん}だやつらは \ruby[g]{気}{き}の \ruby[g]{毒}{どく}だが
 \ruby[g]{三十年}{さんじゅうねん}の \ruby[g]{間}{あいだ}には
たとへいくさに \ruby[g]{行}{い}かなくたつて
 \ruby[g]{一割}{いちわり}ぐら \ruby[g]{ゐ}{い}は \ruby[g]{死}{し}ぬんぢやないか。
そこでラユーのむかしのともよ
またこどもらよきやうだいよ
 \ruby[g]{北守}{ほくしゅ} \ruby[g]{将軍}{しょうぐん}ソンバーユーと
その \ruby[g]{軍勢}{ぐんぜい}が \ruby[g]{帰}{かえ}つたのだ
 \ruby[g]{門}{もん}をあけてもいゝではないか。」
 

  さあ \ruby[g]{城壁}{じょうへき}のこつちでは、 \ruby[g]{沸}{わ}きたつやうな \ruby[g]{騒動}{そうどう}だ。うれしまぎれに \ruby[g]{泣}{な}くものや、 \ruby[g]{両手}{りょうて}をあげて \ruby[g]{走}{はし}るもの、じぶんで \ruby[g]{門}{もん}をあけようとして、 \ruby[g]{番兵}{ばんぺい}たちに \ruby[g]{叱}{しか}られるもの、もちろん \ruby[g]{王}{おう}のお \ruby[g]{宮}{みや}へは \ruby[g]{使}{つか}が \ruby[g]{急い}{いそ}で \ruby[g]{走}{はし}つて \ruby[g]{行}{い}き、 \ruby[g]{城門}{じょうもん}の \ruby[g]{扉}{と}はぴしやんと \ruby[g]{開}{あ}いた。おもての \ruby[g]{方}{ほう}の \ruby[g]{兵隊}{へいたい}たちも、もううれしくて、 \ruby[g]{馬}{うま}にすがつて \ruby[g]{泣}{な}いて \ruby[g]{ゐ}{い}る。
   \ruby[g]{顔}{かお}から \ruby[g]{肩}{かた}から \ruby[g]{灰}{はい}いろの、 \ruby[g]{北守}{ほくしゅ} \ruby[g]{将軍}{しょうぐん}ソンバーユーは、わざとくしやくしや \ruby[g]{顔}{かお}をしかめ、しづかに \ruby[g]{馬}{うま}のたづなをとつて、まつすぐを \ruby[g]{向}{む}いて \ruby[g]{先登}{せんとう}に \ruby[g]{立}{た}ち、それからラッパや \ruby[g]{太鼓}{たいこ}の \ruby[g]{類}{るい}、 \ruby[g]{三角}{さんかく}ばたのついた \ruby[g]{槍}{やり}、まつ \ruby[g]{青}{あお}に \ruby[g]{錆}{さ}びた \ruby[g]{銅}{どう}のほこ、それから \ruby[g]{白}{しろ}い \ruby[g]{矢}{や}をしよつた、 \ruby[g]{兵隊}{へいたい}たちが \ruby[g]{入}{はい}つてくる。 \ruby[g]{馬}{うま}は \ruby[g]{太鼓}{たいこ}に \ruby[g]{歩調}{ほちょう}を \ruby[g]{合}{あわ}せ、 \ruby[g]{殊}{こと}にもさきのソン \ruby[g]{将軍}{しょうぐん}の \ruby[g]{白馬}{しろうま}は、 \ruby[g]{歩}{ある}くたんびに \ruby[g]{膝}{ひざ}がぎちぎち \ruby[g]{音}{おと}がして、ちやうどひやうしをとるやうだ。 \ruby[g]{兵隊}{へいたい}たちは \ruby[g]{軍歌}{ぐんか}をうたふ。
「みそかの \ruby[g]{晩}{ばん}とついたちは
 \ruby[g]{砂漠}{さばく}に \ruby[g]{黒}{くろ}い \ruby[g]{月}{つき}が \ruby[g]{立}{た}つ。
 \ruby[g]{西}{にし}と \ruby[g]{南}{みなみ}の \ruby[g]{風}{かぜ}の \ruby[g]{夜}{よる}は
 \ruby[g]{月}{つき}は \ruby[g]{冬}{ふゆ}でもまつ \ruby[g]{赤}{あか}だよ。
 \ruby[g]{雁}{がん}が \ruby[g]{高}{たか}みを \ruby[g]{飛}{と}ぶときは
 \ruby[g]{敵}{てき}が \ruby[g]{遠}{とお}くへ \ruby[g]{遁}{に}げるのだ。
 \ruby[g]{追}{おう}はうと \ruby[g]{馬}{うま}にまたがれば
にはかに \ruby[g]{雪}{ゆき}がどしやぶりだ。」
   \ruby[g]{兵隊}{へいたい}たちは \ruby[g]{進}{すす}んで \ruby[g]{行}{い}つた。 \ruby[g]{九万}{きゅうまん}の \ruby[g]{兵}{へい}といふものはたゞ \ruby[g]{見}{み}ただけでもぐつたりする。
「 \ruby[g]{雪}{ゆき}の \ruby[g]{降}{お}る \ruby[g]{日}{ひ}はひるまでも
そらはいちめんまつくらで
わづかに \ruby[g]{雁}{がん}の \ruby[g]{行}{い}くみちが
ぼんやり \ruby[g]{白}{しろ}く \ruby[g]{見}{み}えるのだ。
 \ruby[g]{砂}{すな}がこごえて \ruby[g]{飛}{と}んできて
 \ruby[g]{枯}{か}れたよもぎをひつこぬく。
 \ruby[g]{抜}{ぬ}けたよもぎは \ruby[g]{次次}{つぎつぎ}と
 \ruby[g]{都}{みやこ}の \ruby[g]{方}{ほう}へ \ruby[g]{飛}{と}んで \ruby[g]{行}{い}く。」
  みんなは、みちの \ruby[g]{両側}{りょうがわ}に、 \ruby[g]{垣}{かき}をきづいて、ぞろつとならび、 \ruby[g]{泪}{なみだ}を \ruby[g]{流}{な}してこれを \ruby[g]{見}{み}た。
  かくて、バーユー \ruby[g]{将軍}{しょうぐん}が、 \ruby[g]{三町}{さんちょう}ばかり \ruby[g]{進}{すす}んで \ruby[g]{行}{い}つて、 \ruby[g]{町}{まち}の \ruby[g]{広場}{ひろば}についたとき、 \ruby[g]{向}{む}ふのお \ruby[g]{宮}{みや}の \ruby[g]{方角}{ほうがく}から、 \ruby[g]{黄}{き}いろな \ruby[g]{旗}{はた}がひらひらして、 \ruby[g]{誰}{たれ}かこつちへやつてくる。これはたしかに \ruby[g]{知}{し}らせが \ruby[g]{行}{い}つて、 \ruby[g]{王}{おう}から \ruby[g]{迎}{むか}ひが \ruby[g]{来}{き}たのである。
  ソン \ruby[g]{将軍}{しょうぐん}は \ruby[g]{馬}{うま}をとめ、ひたひに \ruby[g]{高}{たか}く \ruby[g]{手}{て}をかざし、よくよくそれを \ruby[g]{見}{み}きはめて、それから \ruby[g]{俄}{には}かに \ruby[g]{一礼}{いちれい}し、 \ruby[g]{急}{いそ}いで、 \ruby[g]{馬}{うま}を \ruby[g]{降}{お}りようとした。ところが \ruby[g]{馬}{うま}を \ruby[g]{降}{お}りれない、もう \ruby[g]{将軍}{しょうぐん}の \ruby[g]{両足}{りょうあし}は、しつかり \ruby[g]{馬}{うま}の \ruby[g]{鞍}{くら}につき、 \ruby[g]{鞍}{くら}はこんどは、がつしりと \ruby[g]{馬}{うま}の \ruby[g]{背中}{せなか}にくつついて、もうどうしてもはなれない。さすが \ruby[g]{豪気}{ごうき}の \ruby[g]{将軍}{しょうぐん}も、すつかりあわてて \ruby[g]{赤}{あか}くなり、 \ruby[g]{口}{くち}をびくびく \ruby[g]{横}{よこ}に \ruby[g]{曲}{ま}げ、 \ruby[g]{一}{ひと} \ruby[g]{生}{しょう}けん \ruby[g]{命}{めい}、はね \ruby[g]{下}{お}りようとするのだが、どうにもからだがうごかなかつた。あゝこれこそじつに \ruby[g]{将軍}{しょうぐん}が、 \ruby[g]{三十年}{さんじゅうねん}も、 \ruby[g]{国境}{こっきょう}の \ruby[g]{空気}{くうき}の \ruby[g]{乾}{かわ}いた \ruby[g]{砂漠}{さばく}のなかで、 \ruby[g]{重}{おも}いつとめを \ruby[g]{肩}{かた}に \ruby[g]{負}{お}ひ、 \ruby[g]{一度}{いちど}も \ruby[g]{馬}{うま}を \ruby[g]{下}{お}りないために、 \ruby[g]{馬}{うま}とひとつになつたのだ。おまけに \ruby[g]{砂漠}{さばく}のまん \ruby[g]{中}{なか}で、どこにも \ruby[g]{草}{くさ}の \ruby[g]{生}{は}えるところがなかつたために、 \ruby[g]{多分}{たぶん}はそれが \ruby[g]{将軍}{しょうぐん}の \ruby[g]{顔}{かお}を \ruby[g]{見付}{みつ}けて \ruby[g]{生}{は}えたのだらう。 \ruby[g]{灰}{はい}いろをしたふしぎなものがもう \ruby[g]{将軍}{しょうぐん}の \ruby[g]{顔}{かお}や \ruby[g]{手}{て}や、まるでいちめん \ruby[g]{生}{は}えて \ruby[g]{ゐ}{い}た。 \ruby[g]{兵隊}{へいたい}たちにも \ruby[g]{生}{は}えて \ruby[g]{ゐ}{い}た。そのうち \ruby[g]{使}{つか}ひの \ruby[g]{大臣}{だいじん}は、だんだん \ruby[g]{近}{ちか}くやつて \ruby[g]{来}{き}て、もうまつさきの \ruby[g]{大}{おお}きな \ruby[g]{槍}{やり}や、 \ruby[g]{旗}{はた}のしるしも \ruby[g]{見}{み}えて \ruby[g]{来た}{き  }。
   \ruby[g]{将軍}{しょうぐん}、 \ruby[g]{馬}{うま}を \ruby[g]{下}{お}りなさい。 \ruby[g]{王様}{おうさま}からのお \ruby[g]{迎}{むか}ひです。 \ruby[g]{将軍}{しょうぐん}、 \ruby[g]{馬}{うま}を \ruby[g]{下}{お}りなさい。 \ruby[g]{向}{む}ふの \ruby[g]{列}{れつ}で \ruby[g]{誰}{だれ}か \ruby[g]{云}{い}ふ。 \ruby[g]{将軍}{しょうぐん}はまた \ruby[g]{手}{て}をばたばたしたが、やつぱりからだがはなれない。
  ところが \ruby[g]{迎}{むか}ひの \ruby[g]{大臣}{だいじん}は、 \ruby[g]{鮒}{ふな}よりひどい \ruby[g]{近眼}{きんがん}だつた。わざと \ruby[g]{馬}{うま}から \ruby[g]{下}{お}りないで、 \ruby[g]{両手}{りょうて}を \ruby[g]{振}{ふ}つて、みんなに \ruby[g]{何}{なに}か \ruby[g]{命令}{めいれい}してると \ruby[g]{考}{かんが}へた。
「 \ruby[g]{謀叛}{むほん}だな。よし。 \ruby[g]{引}{ひ}き \ruby[g]{上}{うえ}げろ。」さう \ruby[g]{大臣}{だいじん}はみんなに \ruby[g]{云}{い}つた。そこで \ruby[g]{大臣}{だいじん} \ruby[g]{一行}{いっこう}は、くるつと \ruby[g]{馬}{うま}を \ruby[g]{立}{た}て \ruby[g]{直}{なお}し、 \ruby[g]{黄}{き}いろな \ruby[g]{塵}{ちり}をあげながら、 \ruby[g]{一目}{いちもく} \ruby[g]{散}{さん}に \ruby[g]{戻}{もど}つて \ruby[g]{行}{い}く。ソン \ruby[g]{将軍}{しょうぐん}はこれを \ruby[g]{見}{み}て \ruby[g]{肩}{かた}をすぼめてため \ruby[g]{息}{いき}をつき、しばらくぼんやりして \ruby[g]{ゐ}{い}たが、 \ruby[g]{俄}{にわ}かにうしろを \ruby[g]{振}{ふ}り \ruby[g]{向}{む}いて、 \ruby[g]{軍師}{ぐんし}の \ruby[g]{長}{ちょう}を \ruby[g]{呼}{よ}び \ruby[g]{寄}{よ}せた。
「おまへはすぐに \ruby[g]{鎧}{よろひ}を \ruby[g]{脱}{ぬ}いで、おれの \ruby[g]{刀}{かたな}と \ruby[g]{弓}{ゆみ}をもち、 \ruby[g]{早}{はや}くお \ruby[g]{宮}{みや}へ \ruby[g]{行}{い}つてくれ。それから \ruby[g]{誰}{だれ}かにかう \ruby[g]{云}{い}ふのだ。 \ruby[g]{北守}{ほくしゅ} \ruby[g]{将軍}{しょうぐん}ソンバーユーは、あの \ruby[g]{国境}{こっきょう}の \ruby[g]{砂}{すな}漠の \ruby[g]{上}{うえ}で、 \ruby[g]{三十年}{さんじゅうねん}のひるも \ruby[g]{夜}{よる}も、 \ruby[g]{馬}{うま}から \ruby[g]{下}{お}りるひまがなく、たうとうからだが \ruby[g]{鞍}{くら}につき、そのまた \ruby[g]{鞍}{くら}が \ruby[g]{馬}{うま}について、どうにもお \ruby[g]{前}{まえ}へ \ruby[g]{出}{だ}られません。これからお \ruby[g]{医者}{いしゃ}に \ruby[g]{行}{い}きまして、やがて \ruby[g]{参内}{さんだい}いたします。かうていねいに \ruby[g]{云}{い}つてくれ。」
   \ruby[g]{軍師}{ぐんし}の \ruby[g]{長}{ちょう}はうなづいて、すばやく \ruby[g]{鎧}{よろい}と \ruby[g]{兜}{かぶと}を \ruby[g]{脱}{ぬ}ぎ、ソン \ruby[g]{将軍}{しょうぐん}の \ruby[g]{刀}{かたな}をもつて、 \ruby[g]{一}{ひと} \ruby[g]{目散}{もくさん}にかけて \ruby[g]{行}{い}く。ソン \ruby[g]{将軍}{しょうぐん}はみんに \ruby[g]{云}{うん}つた。
「 \ruby[g]{全軍}{ぜんぐん}しづかに \ruby[g]{馬}{うま}をおり、 \ruby[g]{兜}{かぶと}をぬいで \ruby[g]{地}{ち}に \ruby[g]{座}{すわ}れ。ソン \ruby[g]{大将}{たいしょう}はたゞ \ruby[g]{今}{いま}から、ちよつとお \ruby[g]{医者}{いしゃ}へ \ruby[g]{行}{い}つてくる。そのうち \ruby[g]{音}{おと}をたてないで、じいつとやすんで \ruby[g]{ゐ}{い}てくれい。わかつたか。」
「わかりました。 \ruby[g]{将軍}{しょうぐん}」 \ruby[g]{兵隊共}{へいたいども}は \ruby[g]{声}{こえ}をそろへて \ruby[g]{一度}{いちど}に \ruby[g]{叫}{さけ}ぶ。 \ruby[g]{将軍}{しょうぐん}はそれを \ruby[g]{手}{て}で \ruby[g]{制}{せい}し、 \ruby[g]{急}{いそ}いで \ruby[g]{馬}{うま}に \ruby[g]{鞭}{むち}うつた。たびたびペたんと \ruby[g]{砂漠}{さばく}に \ruby[g]{寝}{ね}た、この \ruby[g]{有名}{ゆうめい}な \ruby[g]{白馬}{しろうま}は、こゝで \ruby[g]{最後}{さいご}の \ruby[g]{力}{ちから}を \ruby[g]{出}{だ}し、がたがたがたがた \ruby[g]{鳴}{な}りながら、 \ruby[g]{風}{かぜ}より \ruby[g]{早}{はや}くかけ \ruby[g]{出}{だ}した。さて \ruby[g]{将軍}{しょうぐん}は \ruby[g]{十町}{じゅっちょう}ばかり、 \ruby[g]{夢中}{むちゅう}で \ruby[g]{馬}{うま}を \ruby[g]{走}{はし}らせて、 \ruby[g]{大}{おお}きな \ruby[g]{坂}{さか}の \ruby[g]{下}{した}に \ruby[g]{来た}{き  }。それから \ruby[g]{俄}{には}かにかう \ruby[g]{云}{い}つた。
「 \ruby[g]{上手}{じょうず}な \ruby[g]{医者}{いしゃ}はいつたい \ruby[g]{誰}{だれ}だ。」
   \ruby[g]{一人}{ひとり}の \ruby[g]{大工}{だいく}が \ruby[g]{返事}{へんじ}した。
「それはリンパー \ruby[g]{先生}{せんせい}です。」
「そのリンパーはどこに \ruby[g]{居}{お}る。」
「すぐこの \ruby[g]{坂}{さか}のま \ruby[g]{上}{うえ}です。あの \ruby[g]{三}{みっ}つある \ruby[g]{旗}{はた}のうち、 \ruby[g]{一番}{いちばん} \ruby[g]{左}{ひだり}でございます。」
「よろしい、しゆう。」と \ruby[g]{将軍}{しょうぐん}は、 \ruby[g]{例}{れい}の \ruby[g]{白馬}{はくば}に \ruby[g]{一}{ひと} \ruby[g]{鞭}{むち}くれて、 \ruby[g]{一気}{いっき}に \ruby[g]{坂}{さか}をかけあがる。 \ruby[g]{大工}{だいく}はあとでぶつぶつ \ruby[g]{云}{い}つた。
「 \ruby[g]{何}{なん}だ、あいつは \ruby[g]{野蛮}{やばん}なやつだ。ひとからものを \ruby[g]{教}{おそ}はつて、よろしい、しゆう  とはいつたいなんだ。」
  ところがバーユー \ruby[g]{将軍}{しょうぐん}は、そんなことには \ruby[g]{構}{かま}はない。そこらをうろうろあるいて \ruby[g]{ゐ}{い}る、 \ruby[g]{病人}{びょうにん}たちをはね \ruby[g]{越}{こ}えて、 \ruby[g]{門}{もん}の \ruby[g]{前}{まえ}まで \ruby[g]{上}{のぼ}つて \ruby[g]{ゐ}{い}た。なるほど \ruby[g]{門}{もん}のはしらには、 \ruby[g]{小医}{しょうい}リンパー \ruby[g]{先生}{せんせい}と、 \ruby[g]{金看板}{きんかんばん}がかけてある。 
             \ruby[g]{三}{さん}、リンパー \ruby[g]{先生}{せんせい}

  さてソンバーユー \ruby[g]{将軍}{しょうぐん}は、いまやリンパー \ruby[g]{先生}{せんせい}の、 \ruby[g]{大玄関}{おおげんかん}を \ruby[g]{乗}{の}り \ruby[g]{切}{き}つて、どしどし \ruby[g]{廊下}{ろうか}へ \ruby[g]{入}{はい}つて \ruby[g]{行}{い}く。さすがはリンパー \ruby[g]{病院}{びょういん}だ、どの \ruby[g]{天井}{てんじょう}も \ruby[g]{室}{へや}の \ruby[g]{扉}{と}も、 \ruby[g]{高}{たか}さが \ruby[g]{二丈}{にじょう}ぐら \ruby[g]{ゐ}{い}ある。
「 \ruby[g]{医者}{いしゃ}はどこかね。 \ruby[g]{診}{み}てもらひたい。」ソン \ruby[g]{将軍}{しょうぐん}は \ruby[g]{号}{ごう} \ruby[g]{令}{れい}した。
「あなたは \ruby[g]{一体}{いったい} \ruby[g]{何}{なん}ですか。 \ruby[g]{馬}{うま}のまんまで \ruby[g]{入}{はい}るとは、あんまり \ruby[g]{乱暴}{らんぼう}すぎませう。」 \ruby[g]{萌黄}{もえぎ}の \ruby[g]{長}{おさ}い \ruby[g]{服}{ふく}を \ruby[g]{着}{き}て、 \ruby[g]{頭}{あたま}を \ruby[g]{剃}{そ}つた \ruby[g]{一人}{ひとり}の \ruby[g]{弟子}{でし}が、 \ruby[g]{馬}{うま}のくつわをつかまへた。
「おまへが \ruby[g]{医者}{いしゃ}のリンパーか、 \ruby[g]{早}{はや}くわが \ruby[g]{輩}{はい}の \ruby[g]{病気}{びょうき}を \ruby[g]{診}{み}ろ。」
「いゝえ、リンパー \ruby[g]{先生}{せんせい}は、 \ruby[g]{向}{むこ}ふの \ruby[g]{室}{へや}に \ruby[g]{居}{お}られます。けれどもご \ruby[g]{用}{よう}がおありなら、 \ruby[g]{馬}{うま}から \ruby[g]{下}{お}りていたゞきたい。」
「いゝや、そいつができんのぢや。 \ruby[g]{馬}{うま}からすぐに \ruby[g]{下}{お}りれたら、 \ruby[g]{今}{いま}ごろはもう \ruby[g]{王様}{おうさま}の、 \ruby[g]{前}{まえ}へ \ruby[g]{行}{い}つてた \ruby[g]{筈}{はず}なんぢや。」
「ははあ、 \ruby[g]{馬}{うま}から \ruby[g]{降}{お}りられない。そいつは \ruby[g]{脚}{あし}の \ruby[g]{硬直}{こうちょく}だ。そんならいゝです。おいでなさい。」
   \ruby[g]{弟子}{でし}は \ruby[g]{向}{む}ふの \ruby[g]{扉}{とびら}をあけた。ソン \ruby[g]{将軍}{しょうぐん}はぱかぱかと \ruby[g]{馬}{うま}を \ruby[g]{鳴}{な}らしてはひつて \ruby[g]{行}{い}つた。 \ruby[g]{中}{なか}には \ruby[g]{人}{ひと}がいつぱいで、そのまん \ruby[g]{中}{なか}に \ruby[g]{先生}{せんせい}らしい、 \ruby[g]{小}{ちい}さな \ruby[g]{人}{ひと}が \ruby[g]{床几}{しやうぎ}に \ruby[g]{座}{すわ}り、しきりに \ruby[g]{一人}{ひとり}の \ruby[g]{眼}{め}を \ruby[g]{診}{み}て \ruby[g]{ゐ}{い}る。
「ひとつこつちをたのむのぢや。 \ruby[g]{馬}{うま}から \ruby[g]{降}{お}りられないでなう。」さう \ruby[g]{将軍}{しょうぐん}はやさしく \ruby[g]{云}{うん}つた。ところがリンパー \ruby[g]{先生}{せんせい}は、 \ruby[g]{見向}{みむ}きもしないし \ruby[g]{動}{うご}きもしない。やつぱりじつと \ruby[g]{眼}{め}を \ruby[g]{見}{み}て \ruby[g]{ゐ}{い}る。
「おい、きみ、 \ruby[g]{早}{はや}くこつちを \ruby[g]{見}{み}んか。」 \ruby[g]{将軍}{しょうぐん}が \ruby[g]{怒鳴}{どな}り \ruby[g]{出}{だ}したので、 \ruby[g]{病人}{びょうにん}たちはびくつとした。ところが \ruby[g]{弟子}{でし}がしづかに \ruby[g]{云}{うん}つた。
「 \ruby[g]{診}{み}るには \ruby[g]{番}{ばん}がありますからな。あなたは \ruby[g]{九十六番}{くじゅうろくばん}で、いまは \ruby[g]{六人目}{ろくにんめ}ですから、もう \ruby[g]{九十人}{くじゅうにん}お \ruby[g]{待}{ま}ちなさい。」
「 \ruby[g]{黙}{だま}れ、きさまは \ruby[g]{我輩}{わがはい}に、 \ruby[g]{七}{しち} \ruby[g]{十二人待}{じゅうににんま}てつと \ruby[g]{云}{い}ふか。おれを \ruby[g]{誰}{だれ}だと \ruby[g]{考}{かんが}へる。 \ruby[g]{北守}{ほくしゅ} \ruby[g]{将軍}{しょうぐん}ソンバーユーだ。 \ruby[g]{九万人}{きゅうまんにん}もの \ruby[g]{兵隊}{へいたい}を、 \ruby[g]{町}{まち}の \ruby[g]{広場}{ひろば}に \ruby[g]{待}{ま}たせてある。おれが \ruby[g]{一人}{ひとり}を \ruby[g]{待}{ま}つことは \ruby[g]{七万二千}{しちまんにせん}の \ruby[g]{兵隊}{へいたい}が、 \ruby[g]{向}{む}ふの \ruby[g]{方}{ほう}で \ruby[g]{待}{ま}つことだ。すぐ \ruby[g]{見}{み}ないならけちらすぞ。」 \ruby[g]{将軍}{しょうぐん}はもう \ruby[g]{鞭}{むち}をあげ \ruby[g]{馬}{うま}は \ruby[g]{一}{ひと}いきはねあがり、 \ruby[g]{病人}{びょうにん}たちは \ruby[g]{泣}{な}きだした。ところがリンパー \ruby[g]{先生}{せんせい}は、やつぱりびくともして \ruby[g]{ゐ}{い}ない、てんでこつちを \ruby[g]{見}{み}もしない。その \ruby[g]{先生}{せんせい}の \ruby[g]{右手}{みぎて}から、 \ruby[g]{黄}{き}の \ruby[g]{綾}{あや}を \ruby[g]{着}{き}た \ruby[g]{娘}{むすめ}が \ruby[g]{立}{た}つて、 \ruby[g]{花瓶}{かびん}にさした \ruby[g]{何}{なに}かの \ruby[g]{花}{はな}を、 \ruby[g]{一枝}{ひとえだ}とつて \ruby[g]{水}{みず}につけ、やさしく \ruby[g]{馬}{うま}につきつけた。 \ruby[g]{馬}{うま}はぱくつとそれを \ruby[g]{噛}{か}み、 \ruby[g]{大}{おお}きな \ruby[g]{息}{いき}を \ruby[g]{一}{ひと}つして、ぺたんと \ruby[g]{四}{よつ}つ \ruby[g]{脚}{あし}を \ruby[g]{折}{お}り、 \ruby[g]{今度}{こんど}はごうごういびきをかいて、 \ruby[g]{首}{くび}を \ruby[g]{落}{おと}してねむつてしまふ。ソン \ruby[g]{将軍}{しょうぐん}はまごついた。
「あ、 \ruby[g]{馬}{うま}のやつ、 \ruby[g]{又}{また} \ruby[g]{参}{まい}つたな。 \ruby[g]{困}{こん}つた。 \ruby[g]{困}{こん}つた。 \ruby[g]{困}{こん}つた。」と \ruby[g]{云}{い}つて、 \ruby[g]{急い}{いそ}で \ruby[g]{鎧}{よろひ}のかくしから、 \ruby[g]{塩}{しお}の \ruby[g]{袋}{ふくろ}をとりだして、 \ruby[g]{馬}{うま}に \ruby[g]{喰}{た}べさせようとする。
「おい、 \ruby[g]{起}{お}きんかい。あんまり \ruby[g]{情}{なさ}けないやつだ。あんなにひどく \ruby[g]{難儀}{なんぎ}して、やつと \ruby[g]{都}{みやこ}に \ruby[g]{帰}{かえ}つて \ruby[g]{来}{き}ると、すぐ \ruby[g]{気}{き}がゆるんで \ruby[g]{死}{し}ぬなんて、ぜんたいどういふ \ruby[g]{考}{かんがえ}なのか。こら、 \ruby[g]{起}{お}きんかい。 \ruby[g]{起}{お}きんかい。しつ、ふう、どう、おい、この \ruby[g]{塩}{しお}を、ほんの \ruby[g]{一口}{ひとくち}たべんかい。」それでも \ruby[g]{馬}{うま}は、やつぱりぐうぐうねむつて \ruby[g]{ゐ}{い}る。ソン \ruby[g]{将軍}{しょうぐん}はたうとう \ruby[g]{泣}{な}いた。
「おい、きみ、わしはとにかくに、 \ruby[g]{馬}{うま}だけどうかみてくれたまへ。こいつは \ruby[g]{北}{きた}の \ruby[g]{国境}{こっきょう}で、 \ruby[g]{三十年}{さんじゅうねん}もはたらいたのだ。」
  むすめはだまつて \ruby[g]{笑}{わらい}つて \ruby[g]{ゐ}{い}たが、このときリンパー \ruby[g]{先生}{せんせい}が、いきなりこつちを \ruby[g]{振}{ふ}り \ruby[g]{向}{む}いて、まるで \ruby[g]{将軍}{しょうぐん}の \ruby[g]{胸底}{むなぞこ}から、 \ruby[g]{馬}{うま}の \ruby[g]{頭}{あたま}も \ruby[g]{見徹}{みとお}すやうな、するどい \ruby[g]{眼}{め}をしてしづかに \ruby[g]{云}{い}つた。
「 \ruby[g]{馬}{うま}はまもなく \ruby[g]{治}{なお}ります。あなたの \ruby[g]{病気}{びょうき}をしらべるために、 \ruby[g]{馬}{うま}を \ruby[g]{座}{すわ}らせただけです。あなたはそれで \ruby[g]{向}{む}ふの \ruby[g]{方}{ほう}で、 \ruby[g]{何}{なに}か \ruby[g]{病気}{びょうき}をしましたか。」
「いゝや、 \ruby[g]{病気}{びょうき}はしなかつた。 \ruby[g]{病気}{びょうき}は \ruby[g]{別}{べつ}にしなかつたが、 \ruby[g]{狐}{きつね}のために \ruby[g]{欺}{だま}されて、どうもときどき \ruby[g]{困}{こん}つたぢや。」
「それは、どういふ
 \ruby[g]{風}{ふう}ですか。」
「 \ruby[g]{向}{む}ふの \ruby[g]{狐}{きつね}はいかんのぢや。 \ruby[g]{十万}{じゅうまん} \ruby[g]{近}{ちか}い \ruby[g]{軍勢}{ぐんぜい}を、たゞ \ruby[g]{一}{ひと}ぺんに \ruby[g]{欺}{だま}すんぢや。 \ruby[g]{夜}{よる}に \ruby[g]{沢山}{たくさん} \ruby[g]{火}{ひ}をともしたり、 \ruby[g]{昼間}{ひるま}いきなり \ruby[g]{破漠}{さばく}の \ruby[g]{上}{うえ}に、大きな \ruby[g]{海}{うみ}をこしらへて、 \ruby[g]{城}{しろ}や \ruby[g]{何}{なん}かも \ruby[g]{出}{だ}したりする。 \ruby[g]{全}{まった}くたちが \ruby[g]{悪}{わる}いんぢや。」
「それを \ruby[g]{狐}{きつね}がしますのですか。」
「 \ruby[g]{狐}{きつね}とそれから、 \ruby[g]{砂鶻}{さこつ}ぢやね、 \ruby[g]{砂鶻}{さこつ}というて \ruby[g]{鳥}{とり}なんぢや。こいつは \ruby[g]{人}{ひと}の \ruby[g]{居}{お}らないときは、 \ruby[g]{高}{たか}い \ruby[g]{処}{ところ}を \ruby[g]{飛}{と}んで \ruby[g]{ゐ}{い}て、 \ruby[g]{誰}{だれ}かを \ruby[g]{見}{み}ると \ruby[g]{試}{ため}しに \ruby[g]{来}{き}る。 \ruby[g]{馬}{うま}のしつぽを抜いたりね。 \ruby[g]{目}{め}をねらつたりするもんで、こいつがでたらもう \ruby[g]{馬}{うま}は、がたがたふるへてようあるかんね。」
「そんなら \ruby[g]{一}{ひと}ペん \ruby[g]{欺}{だま}されると、 \ruby[g]{何日}{なんにち}ぐら \ruby[g]{ゐ}{い}でよくなりますか。」
「まあ \ruby[g]{四日}{よっか}ぢやね。 \ruby[g]{五日}{いつか}のときもあるやうぢや。」
「それであなたは \ruby[g]{今}{いま}までに、 \ruby[g]{何}{なん}べんぐら \ruby[g]{ゐ}{い} \ruby[g]{欺}{だま}されました?」
「ごく少くて \ruby[g]{十}{じゅっ}ぺんぢやらう。」
「それではお \ruby[g]{尋}{たず}ねいたします。 \ruby[g]{百}{ひゃく}と \ruby[g]{百}{ひゃく}とを \ruby[g]{加}{くわ}へると \ruby[g]{答}{こたえ}はいくらになりますか。」
「 \ruby[g]{百八十}{ひゃくはちじゅう}ぢや。」
「それでは \ruby[g]{二百}{にひゃく}と二 \ruby[g]{百}{ひゃく}では。」
「さやう、 \ruby[g]{三百六十}{さんびゃくろくじゅう}だらう。」
「そんならも \ruby[g]{一}{ひと}つ \ruby[g]{伺}{うかが}ひますが、 \ruby[g]{十}{じゅっ}の \ruby[g]{二倍}{にばい}は \ruby[g]{何}{いか}ほどですか。」
「それはもちろん \ruby[g]{十八}{じゅうはち}ぢや。」
「なるほど、すつかりわかりました。あなたは \ruby[g]{今}{いま}でもまだ \ruby[g]{少}{すこ}し、 \ruby[g]{砂漠}{さばく}のためにつかれて \ruby[g]{ゐ}{い}ます。つまり \ruby[g]{十}{じゅっ}パーセントです。それではなほしてあげませう。」
  パー \ruby[g]{先生}{せんせい}は \ruby[g]{両手}{りょうて}をふつて、 \ruby[g]{弟子}{でし}にしたくを \ruby[g]{云}{い}ひ \ruby[g]{付}{つ}けた。 \ruby[g]{弟子}{でし}は大きな \ruby[g]{銅鉢}{どうばち}に、 \ruby[g]{何}{なに}かの \ruby[g]{薬}{くすり}をいつぱい \ruby[g]{盛}{も}つて、 \ruby[g]{布巾}{ふきん}を \ruby[g]{添}{そ}へて \ruby[g]{持}{も}つて \ruby[g]{来た}{き  }。ソン \ruby[g]{将軍}{しょうぐん}は \ruby[g]{両手}{りょうて}を \ruby[g]{出}{だ}して \ruby[g]{鉢}{はち}をきちんと \ruby[g]{受}{う}けとつた。パー \ruby[g]{先生}{せんせい}は \ruby[g]{片袖}{かたそで}まくり、 \ruby[g]{布巾}{ふきん}に \ruby[g]{薬}{くすり}をいつぱいひたし、かぶとの \ruby[g]{上}{うえ}からざぶざぶかけて、 \ruby[g]{両手}{りょうて}でそれをゆすぶると、 \ruby[g]{兜}{かぶと}はすぐにすぱりととれた。 \ruby[g]{弟子}{でし}がも \ruby[g]{一人}{ひとり}、もひとつ \ruby[g]{別}{べつ}の \ruby[g]{銅}{どう} \ruby[g]{鉢}{はち}へ、 \ruby[g]{別}{べつ}の \ruby[g]{薬}{くすり}をもつてきた。そこでリンパー \ruby[g]{先生}{せんせい}は、 \ruby[g]{別}{べつ}の \ruby[g]{薬}{くすり}でじやぶじやぶ \ruby[g]{洗}{あら}ふ。 \ruby[g]{雫}{しづく}はまるでまつ \ruby[g]{黒}{くろ}だ。ソン \ruby[g]{将軍}{しょうぐん}は \ruby[g]{心配}{しんぱい}さうに、うつむいたまゝ \ruby[g]{訊}{き}いて \ruby[g]{ゐ}{い}る。
「どうかね、 \ruby[g]{馬}{うま}は \ruby[g]{大丈夫}{だいじょうぶ}かね。」
「もうぢきです。」とパー \ruby[g]{先生}{せんせい}は、つゞけてじやぶじやぶ \ruby[g]{洗}{あら}つて \ruby[g]{ゐ}{い}る。 \ruby[g]{雫}{しずく}がだんだん \ruby[g]{茶}{ちゃ}いろになつて、それからうすい \ruby[g]{黄}{き}いろになつた。それからたうとうもう \ruby[g]{色}{いろ}もなく、ソン \ruby[g]{将軍}{しょうぐん}の \ruby[g]{白髪}{しらが}は、 \ruby[g]{熊}{くま}より \ruby[g]{白く}{しろ} \ruby[g]{輝}{かがや}いた。そこでリンパー \ruby[g]{先生}{せんせい}は、 \ruby[g]{布巾}{ふきん}を \ruby[g]{捨}{す}てて \ruby[g]{両手}{りょうて}を \ruby[g]{洗}{あら}ひ、 \ruby[g]{弟子}{でし}は \ruby[g]{頭}{あたま}と \ruby[g]{顔}{かお}を \ruby[g]{拭}{ふ}く。 \ruby[g]{将軍}{しょうぐん}はぶるつと \ruby[g]{身}{み}ぶるひして、 \ruby[g]{馬}{うま}にきちんと \ruby[g]{起}{お}きあがる。
「どうです、せいせいしたでせう。ところで \ruby[g]{百}{ひゃく}と \ruby[g]{百}{ひゃく}とをたすと、 \ruby[g]{答}{こたえ}はいくらになりますか。」
「もちろんそれは \ruby[g]{二百}{にひゃく}だらう。」
「そんなら \ruby[g]{二百}{にひゃく}と二 \ruby[g]{百}{ひゃく}とたせば。」
「さやう、 \ruby[g]{四百}{しひゃく}にちがひない。」
「 \ruby[g]{十}{じゅう}の \ruby[g]{二倍}{にばい}はどれだけですか。」
「それはもちろん \ruby[g]{二十}{にじゅう}ぢやな。」さつきのことは \ruby[g]{忘}{わす}れた \ruby[g]{風}{ふう}で、ソン \ruby[g]{将軍}{しょうぐん}はけろりと \ruby[g]{云}{い}ふ。
「すつかりおなほりなりました。つまり \ruby[g]{頭}{あたま}の \ruby[g]{目}{め}がふさがつて、 \ruby[g]{一割}{いちわり}いけなかつたのですな。」
「いやいや、わしは \ruby[g]{勘定}{かんじょう}などの、 \ruby[g]{十}{じゅっ}や \ruby[g]{二十}{にじゅう}はどうでもいいんぢや。それは \ruby[g]{算師}{さんし}がやるでなう。わしは \ruby[g]{早速}{さっそく}この \ruby[g]{馬}{うま}と、わしをはなしてもらひたいんぢや。」
「なるほどそれはあなたの \ruby[g]{足}{あし}を、あなたの \ruby[g]{服}{ふく}と \ruby[g]{引}{ひ}きはなすのは、すぐ \ruby[g]{私}{わたし}に \ruby[g]{出}{だ} \ruby[g]{来}{き}るです。いやもう \ruby[g]{離}{はな}れて \ruby[g]{ゐ}{い}る \ruby[g]{筈}{はず}です。けれども、ずぼんが \ruby[g]{鞍}{くら}につき、 \ruby[g]{鞍}{くら}がまた \ruby[g]{馬}{うま}についたのを、はなすといふのは \ruby[g]{別}{べつ}ですな。それはとなりで、 \ruby[g]{私}{わたし}の \ruby[g]{弟}{おとうと}がやつて \ruby[g]{ゐ}{い}ますから、そつちへおいでいただきます。それにいつたいこの \ruby[g]{馬}{うま}もひどい \ruby[g]{病気}{びょうき}にかかつて \ruby[g]{ゐ}{い}ます。」
「そんならわしの \ruby[g]{顔}{かお}から生えた、このもじやもじやはどうぢやらう。」
「そちらもやつぱり \ruby[g]{向}{む}ふです。とにかくひとつとなりの \ruby[g]{方}{ほう}へ、 \ruby[g]{弟子}{でし}をお \ruby[g]{供}{とも}に \ruby[g]{出}{だ}しませう。」
「それではそつちへ \ruby[g]{行}{い}くとしよう。ではさやうなら。」
  さつきの \ruby[g]{白}{しろ}いきものをつけた、むすめが \ruby[g]{馬}{うま}の \ruby[g]{右}{みぎ} \ruby[g]{耳}{みみ}に、 \ruby[g]{息}{いき}を \ruby[g]{一}{ひと}つ吹き \ruby[g]{込}{こ}んだ。 \ruby[g]{馬}{うま}はがばつとはねあがり、ソン \ruby[g]{将軍}{しょうぐん}は \ruby[g]{俄}{には}かに \ruby[g]{背}{せい}が \ruby[g]{高}{たか}くなる、 \ruby[g]{将軍}{しょうぐん}は \ruby[g]{馬}{うま}のたづなをとり、 \ruby[g]{弟子}{でし}とならんで \ruby[g]{室}{へや}を \ruby[g]{出}{だ}る。それから \ruby[g]{庭}{にわ}をよこぎつて \ruby[g]{厚}{あつ}い \ruby[g]{土塀}{どべい}の \ruby[g]{前}{まえ}に \ruby[g]{来}{き}た。小さな \ruby[g]{潜}{くぐ}りがあいて \ruby[g]{ゐ}{い}る。
「いま \ruby[g]{裏門}{うらもん}をあけさせませう。」 \ruby[g]{助手}{じょしゅ}は \ruby[g]{潜}{くぐ}りを \ruby[g]{入}{はい}つて \ruby[g]{行}{い}く。
「いゝや、それには \ruby[g]{及}{およ}ばない。わたしの \ruby[g]{馬}{うま}はこれぐら \ruby[g]{ゐ}{い}、まるで \ruby[g]{何}{なん}とも \ruby[g]{思}{おも}つてやしない。」
   \ruby[g]{将軍}{しょうぐん}は \ruby[g]{馬}{うま}にむちをやる。
  ぎつ、ばつ、ふう。 \ruby[g]{馬}{うま}は \ruby[g]{土塀}{どべい}をはね \ruby[g]{越}{こ}えて、となりのリンプー \ruby[g]{先生}{せんせい}の、けしのはたけをめちやくちやに、 \ruby[g]{踏}{ふ}みつけながら \ruby[g]{立}{た}つて \ruby[g]{ゐ}{い}た。

 
            四、馬医リンブー先生

  ソン将軍が、お医者の弟子と、けしの畑をふみつけて向ふの方へ歩いて行くと、もうあつちからもこつちからも、ぶるるるふうといふやうな、馬の仲間の声がする。そして二人が正面の、 \ruby[g]{巨}{おほ}きな \ruby[g]{棟}{むね}にはひつて行くと、もう四方から馬どもが、二十 \ruby[g]{疋}{ぴき}もかけて来て、 \ruby[g]{蹄}{ひづめ}をことこと鳴らしたり、頭をぶらぶらしたりして、将軍の馬に \ruby[g]{挨拶}{あいさつ}する。
  向ふでリンプー先生は、首のまがつた茶いろの馬に、白い薬を塗つてゐる。さつきの弟子が進んで行つて、ちよつと何かをさゝやくと、馬医のリンプー先生は、わらつてこつちをふりむいた。巨きな鉄の \ruby[g]{胸甲}{むなあて}を、がつしりはめてゐることは、ちやうどやつぱり \ruby[g]{鎧}{よろひ}のやうだ。馬にけられぬためらしい。将軍はすぐその前へ、じぶんの馬を乗りつけた。
「あなたがリンプー先生か。わしは将軍ソンバーユーぢや。何分ひとつたのみたい。」
「いや、その由を \ruby{伺}{うかが}ひました。あなたのお馬はたしか三十九ぐらゐですな。」
「四捨五入して、さうぢや、やつぱり三十九ぢやな。」
「ははあ、たゞいま手術いたします。あなたは馬の上に居て、すこし煙いかしれません。それをご承知くださいますか。」
「煙い?  なんのどうして \ruby{煙}{けむ}ぐらゐ、 \ruby[g]{砂漠}{さばく}で風の吹くときは、一分間に四十五以上、馬を跳躍させるんぢや。それを三つも、やすんだら、もう頭まで埋まるんぢや。」
「ははあ、それではやりませう。おい、フーシユ。」プー先生は弟子を呼ぶ。弟子はおじぎを一つして、小さな \ruby[g]{壺}{つぼ}をもつて来た。プー先生は \ruby[g]{蓋}{ふた}をとり、何か茶いろな薬を出して、馬の \ruby[g]{眼}{まなこ}に塗りつけた。それから「フーシユ」とまた呼んだ。弟子はおじぎを一つして、となりの \ruby[g]{室}{へや}へ入つて行つて、しばらくごとごとしてゐたが、まもなく赤い小さな \ruby[g]{餅}{もち}を、 \ruby[g]{皿}{さら}にのつけて帰つて来た。先生はそれをつまみあげ、しばらく指ではさんだり、 \ruby[g]{匂}{にほひ}をかいだりしてゐたが、何か決心したらしく、馬にぱくりと喰べさせた。ソン将軍は、その \ruby[g]{白馬}{しろうま}の上に居て、待ちくたびれてあくびをした。すると \ruby[g]{俄}{には}かに \ruby[g]{白馬}{しろうま}は、がたがたがたがたふるへ出しそれからからだ一面に、あせとけむりを噴き出した。プー先生はこはさうに、遠くへ行つてながめてゐる。がたがたがたがた鳴りながら、馬はけむりをつゞけて噴いた。そのまた煙が \ruby[g]{無暗}{むやみ}に \ruby[g]{辛}{から}い。ソン将軍も、はじめは我慢してゐたが、たうとう両手を眼にあてて、ごほんごほんとせきをした。そのうちだんだんけむりは消えてこんどは、汗が滝よりひどくながれだす。プー先生は近くへよつて、両手をちよつと \ruby[g]{鞍}{くら}にあて、二つつばかりゆすぶつた。
  たちまち鞍はすぱりとはなれ、はずみを食つた将軍は、床にすとんと落された。ところがさすが将軍だ。いつかきちんと立つてゐる。おまけに鞍と将軍も、もうすつかりとはなれてゐて、将軍はまがつた両足を、両手でぱしやぱしや \ruby[g]{叩}{たた}いたし、馬は俄かに荷がなくなつて、さも見当がつかないらしく、せなかをゆらゆらゆすぶつた。するとバーユー将軍はこんどは馬のはうきのやうなしつぽを持つて、いきなりぐつと引つ張つた。すると何やらまつ白な、尾の形した塊が、ごとりと床にころがり落ちた。馬はいかにも軽さうに、いまは全く毛だけになつたしつぼを、ふさふさ振つてゐる。弟子が三人集つて、馬のからだをすつかりふいた。
「もういゝだらう。歩いてごらん。」
  馬はしづかに歩きだす。あんなにぎちぎち \ruby[g]{軋}{きし}んだ \ruby[g]{膝}{ひざ}がいまではすつかり鳴らなくなつた。プー先生は手をあげて、馬をこつちへ呼び戻し、おじぎを一つ将軍にした。
「いや謝しますぢや。それではこれで。」将軍は、急いで馬に鞍を置き、ひらりとそれにまたがれば、そこらあたりの病気の馬は、ひんひん別れの \ruby[g]{挨拶}{あいさつ}をする。ソン将軍は室を出て \ruby[g]{塀}{へい}をひらりと飛び越えて、となりのリンポー先生の、菊のはたけに飛び込んだ。

            五、リンポー先生

  さてもリンポー先生の、草木を治すその \ruby[g]{室}{へや}は、林のやうなものだつた。あらゆる種類の木や花が、そこらいつぱいならべてあつて、どれにもみんな金だの銀の、 \ruby[g]{巨}{おほ}きな札がついてゐる。そこを、バーユー将軍は、馬から下りて、ゆつくりと、ポー先生の前へ行く。さつきの弟子がさきまはりして、すつかり \ruby[g]{談}{はな}してゐたらしく、ポー先生は薬の \ruby[g]{函}{はこ}と大きな赤い \ruby[g]{団扇}{うちは}をもつて、ごくうやうやしく待つてゐた。ソン将軍は手をあげて、
「これぢや。」と顔を指さした。ポー先生は黄いろな粉を、薬函から取り出して、ソン将軍の顔から肩へ、もういつぱいにふりかけて、それから例のうちはをもつて、ばたばたばたばた扇ぎ出す。するとたちまち、将軍の、顔ぢゆうの毛はまつ赤に変り、みんなふはふは飛び出して、見てゐるうちに将軍は、すつかり顔がつるつるなつた。じつにこのとき将軍は、三十年ぶりにつこりした。
「それではこれで行きますぢや。からだもかるくなつたでなう。」もう将軍はうれしくて、はやてのやうに室を出て、おもての馬に飛び乗れば、馬はたちまち病院の、巨きな門を外に出た。あとから弟子が六人で、兵隊たちの顔から生えた灰いろの毛をとるために、薬の袋とうちはをもつて、ソン将軍を追ひかけた。

            六、北守将軍 \ruby[g]{仙人}{せんにん}となる

  さてソンバーユー将軍は、ポー先生の玄関を、光のやうに飛び出して、となりのリンプー病院を、はやてのごとく通り過ぎ、次のリンパー病院を、斜めに見ながらもう一散に、さつきの坂をかけ下りる。馬は五倍も速いので、もう向ふには兵隊たちの、やすんでゐるのが見えてきた。兵隊たちは心配さうにこつちの方を見てゐたのだが、思はず歓呼の声をあげ、みんな一緒に立ちあがる。そのときお宮の方からはさつきの使ひの軍師の長が一目散にかけて来た。
「あゝ、王様は、すつかりおわかりなりました。あなたのことをおききになつて、おん涙さへ浮べられ、お \ruby[g]{出}{い}でをお待ちでございます。」
  そこへさつきの弟子たちが、薬をもつてやつてきた。兵隊たちはよろこんで、粉をふつてはばたばた扇ぐ。そこで九万の軍隊は、もう \ruby[g]{輪廓}{りんくわく}もはつきりなつた。
  将軍は高く号令した。
「馬にまたがり、気をつけいつ。」
  みんなが馬にまたがれば、まもなくそこらはしんとして、たつた二疋の遅れた馬が、鼻をぶるつと鳴らしただけだ。
「前へ進めつ。」太鼓も \ruby[g]{銅鑼}{どら}も鳴り出して、軍は粛々行進した。
  やがて九万の兵隊は、お宮の前の一里の庭に \ruby[g]{縦横}{じゆうわう}ちやうど三百人、四角な陣をこしらへた。
  ソン将軍は馬を降り、しづかに壇をのぼつて行つて床に額をすりつけた。王はしづかに \ruby[g]{斯}{か}ういつた。
「じつに永らくご苦労だつた。これからはもうこゝに居て、大将たちの大将として、なほ忠勤をはげんでくれ。」
  北守将軍ソンバーユーは涙を垂れてお答へした。
「おことばまことに \ruby{畏}{かしこ}くて、何とお答へいたしていゝか、とみに言葉も \ruby[g]{出}{い}でませぬ。とは云へいまや私は、生きた骨ともいふやうな、役に立たずでございます。 \ruby[g]{砂漠}{さばく}の中に居ました間、どこから敵が見てゐるか、あなどられまいと考へて、いつでもりんと胸を張り、眼を見開いて居りましたのが、いま王様のお前に出て、おほめの \ruby[g]{詞}{ことば}をいたゞきますと、 \ruby[g]{俄}{には}かに眼さへ見えぬやう。背骨も曲つてしまひます。 \ruby[g]{何卒}{なにとぞ}これでお暇を願ひ、郷里に帰りたうございます。」
「それでは \ruby{誰}{だれ}かおまへの代り、大将五人の名を挙げよ。」
  そこでバーユー将軍は、大将四人の名をあげた。そして残りの一人の代り、リン兄弟の三人を国のお医者におねがひした。王は早速許されたので、その場でバーユー将軍は、 \ruby[g]{鎧}{よろひ}もぬげば \ruby[g]{兜}{かぶと}もぬいで、かさかさ薄い麻を着た。そしてじぶんの生れた村のス \ruby[g]{山}{ざん}の \ruby[g]{麓}{ふもと}へ帰つて行つて、 \ruby[g]{粟}{あは}をすこうし \ruby[g]{播}{ま}いたりした。それから粟の間引きもやつた。けれどもそのうち将軍は、だんだんものを食はなくなつてせつかくじぶんで播いたりした、粟も一口たべただけ、水をがぶがぶ \ruby[g]{呑}{の}んでゐた。ところが秋の終りになると、水もさつぱり呑まなくなつて、ときどき空を見上げては何かしやつくりするやうなきたいな形をたびたびした。
  そのうちいつか将軍は、どこにも形が見えなくなつた。そこでみんなは将軍さまは、もう \ruby[g]{仙人}{せんにん}になつたと云つて、ス山の山のいたゞきへ小さなお堂をこしらへて、あの \ruby[g]{白馬}{しろうま}は神馬に祭り、あかしや粟をさゝげたり、麻ののぼりをたてたりした。
  けれどもこのとき国手になつた例のリンパー先生は、会ふ人ごとに斯ういつた。
「どうして、バーユー将軍が、雲だけ食つた \ruby[g]{筈}{はず}はない。おれはバーユー将軍の、からだをよくみて知つてゐる。肺と胃の \ruby[g]{腑}{ふ}は同じでない。きつとどこかの林の中に、お骨があるにちがひない。」なるほどさうかもしれないと思つた人もたくさんあつた。


________________________________________

底本:「新修宮沢賢治全集  第十三巻」筑摩書房
      1980(昭和55)年3月15日初版第1刷発行
      1983(昭和58)年6月30日初版第5刷発行
初出:「児童文学  第一冊」
      1931(昭和6)年7月20日発行
入力:林  幸雄
校正:今井忠夫
2003年9月4日作成
青空文庫作成ファイル:
このファイルは、インターネットの図書館、青空文庫(http://www.aozora.gr.jp/)で作られました。入力、校正、制作にあたったのは、ボランティアの皆さんです。

________________________________________

●表記について
•	このファイルは W3C 勧告 XHTML1.1 にそった形式で作成されています。


\end{document}